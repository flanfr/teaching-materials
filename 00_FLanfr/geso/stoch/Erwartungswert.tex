%%
%% 2020 - 08 -03 φ
%%

\section{Verteilungen}


\subsection{Erwartungswert}\index{Erwartungswert}


Ich schlage Ihnen folgendes Würfelspiel vor.

Sie Würfeln und erhalten dabei die Quadratzahl der erzielten
Würfelpunkte in CHF ausbezahlt. Wenn Sie \zB eine \epsdice{5} werfen,
so erhalten Sie CHF 25.-.

Natürlich müssen Sie einen Grundeinsatz bezahlen, damit Sie bei meinem
Spiel überhaupt mitmachen dürfen.

Drehen wir nun die Rollen einmal um und betrachten Sie sich selbst als
Betreiber dieses «Casinos», also dieses Spiels.

Wie viel CHF soll der Spieler pro Wurf als Einsatz bezahlen, sodass
dieses Spiel mit minimaler Wahrscheinlichkeit besser für den
Casino-Betreiber, also für Sie ausgeht?

\TNT{10}{
  Dazu berechnen wir zuerst das, was wir erwarten, wenn dieses Spiel
  oft gespielt wird. Jedes Würfelresultat wird mit $\frac16$
  Wahrscheinlichkeit auftreten. Somit muss jeder Gewinn ja nur mit
  $\frac16$ Wahrscheinlichkeit ausbezahlt werden.

  Wenn wir nun alle Rückzahlungen mit $\frac16$ multiplizieren und
  diese CHF-Werte zusammenzählen, erhalten wir

  $$\frac16\cdot{} (1 + 4 + 9 + 16 + 25 + 36) = 15.166...$$

  Diese Zahl nennen wir den \textbf{Erwartungswert}.

  In obigem Beispiele ist $\mu = 15.166...$ und die Zufallsvariable
  ordnet jedem Ergebnis (1..6) seinen auszuzahlenden Wert zu.

  Zufallsvariable hier:

  $$X(\omega) = 1.- \textrm{ für } \epsdice{1}; 4.- \textrm{ für }
  \epsdice{2} ; ...$$
  
  Wenn wir als Einsatz CHF 16.- verlangen, wird sich dieses Spiel im
  Endeffekt positiv für das «Casino» auswirken. Wenn wir hingegen nur
  CHF 15.- als Einsatz verlangen, wird früher oder später der Spieler
  mehr auf dem Konto haben.
}%% END TNT
\newpage

\begin{definition}{Zufallsvariable}{}
	Eine Zufallsvariable $X(\omega)$ ordnet jedem Ergebnis eine Zahl
	(typischerweise den Gewinn oder Verlust) zu.
\end{definition}


\begin{definition}{Erwartungswert}{}
	Mit $\mu$ wird der \textbf{Erwartungswert} bezeichnet.
	$\mu$ ist das arithmetische Mittel, das die Zufallsvariable
	annimmt.
\end{definition}
\newpage



\subsection{Wahrscheinlichkeitsverteilungen ...}\index{Wahrscheinlichkeitsverteilung}

Für ein Bernoulli-Experiment können wir uns auch ein Histogramm\index{Histogramm} aufzeichnen.

Nehmen wir wieder unseren Basketballspieler «Basil», der mit Wahrscheinlichkeit von 85\% jeweils trifft. Vorhin hatten wir uns gefragt, wie groß die Wahrscheinlichkeit ist, dass er in 20 Würfen genau 17 Mal [bzw. maximal 17 Mal] trifft.

Dies können wir in einer Wertetabelle und somit auch in einem Histogramm darstellen.
\newpage


\subsubsection{... genaue Anzahl Treffer}
Notieren wir uns, wie groß die Wahrscheinlichkeit ist, dass Basil \textbf{genau} $n$ Mal trifft innerhalb von 20 Würfen:

\begin{tabular}{c|cccccccccccccccc}
  n & 0 & 1 & ... & 8 & 9 & 10   & 11   & 12   & 13  & 14  & 15 & 16 & 17 & 18 & 19 &  20\\
  \%& 0 & 0 & ... & 0 & 0 & 0.02 & 0.11 & 0.46 & 1.6 & 4.5 & 10 & 18 & 24 & 23 & 14 & 3.9
\end{tabular}

Wenn wir diese Zahlen in ein Histogramm eintragen erhalten wir eine
Wahrscheinlichkeitsverteilung:


\bbwCenterGraphic{10cm}{geso/stoch/img/BernoulliSingleBasil.png}


\subsubsection{... kumulierte Anzahl Treffer}
Wie im vorangehenden Kapitel können wir uns die Wahrscheinlichkeiten
auch zusammenzählen und \zB fragen, wie groß ist die
Wahrscheinlichkeit, dass Basil Ballisti (er trifft mit 85\%
Wahrscheinlichkeit) bei 20 Würfen \textbf{maximal} 17
Treffer landet. Dabei darf er für maximal 17 Treffer natürlich auch
weniger Treffer landen. So werden für ein Anzahl Treffer alle
Wahrscheinlichkeiten der weniger erzielten Treffer hinzugezählt. So
erklären sich auch die 100\% um maximal 20 Treffer in 20 Würfen zu
landen.

\begin{tabular}{c|cccccccccccccccc}
  n  & 0 & 1 & ... & 8 & 9 & 10  & 11   & 12   & 13  & 14  & 15 & 16 &  17 & 18 & 19 &  20\\
  \% & 0 & 0 & ... & 0 & 0 &  0.02  &0.13   & 0.59 & 2.2 & 6.7 & 17 & 35 &  60 & 82 & 96 & 100
\end{tabular}

Das entsprechende Diagramm sieht wie folgt aus:

\bbwCenterGraphic{10cm}{geso/stoch/img/BernoulliCumulativeBasil.png}
\newpage



