%%
%% Meta: Master Document
%% Neutral KST Jahr 1
%% Zielgruppe 
%% Welche Zielgruppe soll ausgedruckt werden?
%% Typischerweise entweder TALS oder GESO (nicht beide). TRAINER ist optional für
%% die Trainer Version.
\newif\ifisTALS
\newif\ifisGESO
\newif\ifisTRAINER
\input{zielgruppe_TRAINER}

\input{FL_LayoutDoc}

\renewcommand{\author}{Philipp G. Freimann \& Flavio Lanfranconi}
\renewcommand{\grafikautor}{Ph. G. Freimann}
\renewcommand{\authoremail}{philipp.freimann@bbw.ch}
\renewcommand{\erstellungsdatum}{17. 8. 2022}
\renewcommand{\docversion}{0.8.1 (\LaTeX{})}

\renewcommand{\doctitel}{Mathematik}
\renewcommand{\fachthema}{ Brüche}

%%%%%%%%%%%%%%%%%%%%%%%%%%%%%%%%%%%%%%%%%%%%%%%%%%%%%%%%%%%%%%%%%%%%
%% Gesamt-Skripts benötigen ALINONE (all in one), damit Referenzen auf andere
%% Kapitel funktioniren:
\isALLINONEtrue%%

\scriptStart{}

%% Einstiegsaufgaben
%\input{geso/000/GESO_Einstiegsaufgaben}

%% Algebra I
\input{allg/alg/Bruchterme}


\scriptEnde{}


%% Geometrie I - Plani 2
%% Meta Package Algebra 1 TALS
\part{Arithmetik und Algebra I-2}\index{Arithmetik und Algebra!I|textbf}
\renewcommand{\bbwPartID}{AA1-2}
%\input{allg/alg/Zahlmengen}
%\newpage
%\input{allg/alg/Ordnungsrelationen}
%\input{allg/alg/Betrag}
%\input{allg/alg/Terme}
\input{tals/aa1/Polynome}
%\input{allg/alg/Grundoperationen}
%\input{allg/alg/Binomische_Formeln}
%%weiter in Algebra II
\input{allg/alg/Faktorisieren}
\input{allg/alg/Bruchterme}
%\input{allg/alg/Potenzen} %% Erst im Jahr 2!
\newpage
\input{allg/alg/potenzen_wurzeln/Quadratwurzeln}
\newpage




%% All TALS
%%
% Metapackage TALS Planimetrie
\part{Geometrie I}\index{Planimetrie|textbf}
\renewcommand{\bbwPartID}{PLANI}
\input{tals/trig1/Pythagoras}\newpage
%\input{tals/plani/Vielecke}
%%\input{tals/plani/Elementare_Objekte}\newpage
%%\input{tals/plani/Elemente_und_Zusammenhaenge}\newpage
%%
%% 2019 07 04 Ph. G. Freimann
%%

\section{Kreise}\index{Kreisberechnungen}\index{Berechnungen am Kreis}
\sectuntertitel{Warum sind Seeräuber so schlecht in Geometrie? --- Weil sie $\pi$ raten!}
%%%%%%%%%%%%%%%%%%%%%%%%%%%%%%%%%%%%%%%%%%%%%%%%%%%%%%%%%%%%%%%%%%%%%%%%%%%%%%%%%


\subsection*{Lernziele}

\begin{itemize}
\item einfache Kreisberechnungen
\item Kreisring
\item Tangente
\item Sehne (und Sekante)
\item Segment und Sektor
\end{itemize}

\TadBMTG{56}{4}

\newpage


\subsection{Kreisberührung}\index{Kreisberührung!Planimetrie}


\bbwCenterGraphic{8cm}{tals/plani/img/KleinsterKreis.png}
In obigem Kreis sind zwei kleinere Kreise einbeschrieben. Berechnen
Sie den Radius $k$ des kleinsten Kreises aus dem gegebenen Radius $r$ des
großen Kreises.
\TNT{12}{
  \bbwCenterGraphic{8cm}{tals/plani/img/KleinsterKreisLoesung.png}
  $$\Delta A: x^2 + k^2 = (r-k)^2$$
  $$\Delta B: x^2 + (\frac{r}2 - k)^2 = (\frac{r}2 + k)^2$$
  Ausmultiplizieren und 2. Gleichung von 1. Gleichung subtrahieren:
  $$k^2 - \frac{r^2}4 +rk - k^2  =r^2  -2rk - \frac{r^2}4 - rk$$\
  $$-\frac{r^2}4 + rk = r^2 - 3rk - \frac{r^2}4$$
  $$rk = r^2 - 3rk$$
  $$4k = r$$
  $$k = \frac{r}4$$
}%% END TNT
\newpage


\begin{rezept}{Kreisberührung}{}
  Bei Aufgaben, bei denen sich zwei Kreise \textbf{berühren}, ist es
  von Vorteil, die Mittelpunkte der Kreise mit den Tangentenpunkten zu
  verbinden.

  Danach suchen Sie rechtwinklige Dreiecke.
\end{rezept}

\subsection{Umfang und Fläche}\index{Kreis!Umfang}

\begin{gesetz}{Kreislinie}{}\index{PI@$\pi$ (Pi)}\index{$\pi$}
  Die Länge der Kreislinie wird aus dem Durchmesser $d=2r$  mittels Kreiszahl
  $\pi$ berechnet. Es gilt für den Umfang $U$:
  $$U = 2r\pi = d\pi$$
\end{gesetz}

\begin{gesetz}{Kreisfläche}{}\index{Kreisfläche}
  Die Kreisfläche $A$ eines Kreises mit Radius $r$ wird wie folgt
  berechnet:
  $$A = r^2\pi$$
\end{gesetz}
Herleitung
\TNTeop{Pizza: Schneiden, auslegen = halbe Rechtecksfläche.}

%%%%%%%%%%%%%%%%%%%%%%%%%%%%%%%%%%5
\subsection{Kreisteile}
\begin{gesetz}{Kreisring}{}\index{Kreisring}
  Die Kreisringfläche ist die Differenz der umgebenden Kreisfläche
  (Radius $R$) und
  der inneren Kreisfläche (Radius $r$):

  $$A = A(R) - A(r) = R^2\pi - r^2\pi = (R^2-r^2)\pi$$
\end{gesetz}

\begin{gesetz}{Kreisbogen und
    Kreissektor}{}\index{Kreissektor}\index{Kreisbogen}
  Für den Sektorwinkel $\varphi$ werden Bogen $b$ und Sektorfläche
  $A_{SK}$ wie folgt berechnet:
  $$b = 2r\pi \cdot{}\frac{\varphi}{360\degre} =
  r\pi\cdot{}\frac{\varphi}{180\degre}$$
  $$A_{SK} = r^2\pi\cdot{}\frac{\varphi}{360\degre} = \frac12\cdot{}b\cdot{}r$$
  \end{gesetz}
  
\begin{rezept}{Geometrische Aufgaben}{}
Um geometrische Aufgaben zu lösen, hat bei mir folgendes meist geholfen:

\begin{enumerate}
\item Machen Sie eine Skizze
\item Machen Sie eine möglichst genaue Konstruktion
\item Geben Sie Gegebenem und Gesuchtem Namen
\item Verwenden Sie Farben für Gegebenes
\item Verwenden Sie die selben Farben (od. Symbole) für die selben Streckenlängen, Winkel, Flächen
\item Bei Aufgaben mit Kreisen: Verbinden Sie die Mittelpunkte 
\item Suchen Sie rechtwinklige Dreiecke (Pythagoras)

\end{enumerate}
\end{rezept}  

%Für das Kreissegment Siehe \cite{marthaler20geom} Seite 61 Kap. 4.2.3.
%
%\subsection*{Aufgaben}
%%% Kreisfläche
%%%\TALSAadBFWG{43ff (Kreisfläche)}{159. 160. 165. 168.}
%\AadBMTG{63}{1., 2., 3. und 4.}
%%% Kreissektor und Segment
%%%\TALSAadBFWG{48ff(Kreissektor)}{179. 180. 183. 185. 187. 188. 189. 192. 194.}
%\AadBMTG{65}{12., 13. (Bogenmaß), 14., 15., 17.}
%\AadBMTG{123}{39. a) b) c) d)}%% Aufgabe schon vorgelöst
%%% Kreissegment
%%%\TALSAadBFWG{51ff (Kreissegment)}{197. 198. 200. 204. 206.}
%
%
%%% Vermischte Aufgaben
%%%\TALSAadBFWG{53ff (vermischte Aufgaben)}{209. 215. 219. 223.}
\newpage
%\input{tals/plani/Strahlensaetze}\newpage
%\input{tals/plani/Aehnlichkeit}\newpage





% 2. Jahr
%% Gleichungssysteme
%%
% Metapackage TALS Planimetrie
\part{Geometrie I}\index{Planimetrie|textbf}
\renewcommand{\bbwPartID}{PLANI}
\input{tals/trig1/Pythagoras}\newpage
%\input{tals/plani/Vielecke}
%%\input{tals/plani/Elementare_Objekte}\newpage
%%\input{tals/plani/Elemente_und_Zusammenhaenge}\newpage
%%
%% 2019 07 04 Ph. G. Freimann
%%

\section{Kreise}\index{Kreisberechnungen}\index{Berechnungen am Kreis}
\sectuntertitel{Warum sind Seeräuber so schlecht in Geometrie? --- Weil sie $\pi$ raten!}
%%%%%%%%%%%%%%%%%%%%%%%%%%%%%%%%%%%%%%%%%%%%%%%%%%%%%%%%%%%%%%%%%%%%%%%%%%%%%%%%%


\subsection*{Lernziele}

\begin{itemize}
\item einfache Kreisberechnungen
\item Kreisring
\item Tangente
\item Sehne (und Sekante)
\item Segment und Sektor
\end{itemize}

\TadBMTG{56}{4}

\newpage


\subsection{Kreisberührung}\index{Kreisberührung!Planimetrie}


\bbwCenterGraphic{8cm}{tals/plani/img/KleinsterKreis.png}
In obigem Kreis sind zwei kleinere Kreise einbeschrieben. Berechnen
Sie den Radius $k$ des kleinsten Kreises aus dem gegebenen Radius $r$ des
großen Kreises.
\TNT{12}{
  \bbwCenterGraphic{8cm}{tals/plani/img/KleinsterKreisLoesung.png}
  $$\Delta A: x^2 + k^2 = (r-k)^2$$
  $$\Delta B: x^2 + (\frac{r}2 - k)^2 = (\frac{r}2 + k)^2$$
  Ausmultiplizieren und 2. Gleichung von 1. Gleichung subtrahieren:
  $$k^2 - \frac{r^2}4 +rk - k^2  =r^2  -2rk - \frac{r^2}4 - rk$$\
  $$-\frac{r^2}4 + rk = r^2 - 3rk - \frac{r^2}4$$
  $$rk = r^2 - 3rk$$
  $$4k = r$$
  $$k = \frac{r}4$$
}%% END TNT
\newpage


\begin{rezept}{Kreisberührung}{}
  Bei Aufgaben, bei denen sich zwei Kreise \textbf{berühren}, ist es
  von Vorteil, die Mittelpunkte der Kreise mit den Tangentenpunkten zu
  verbinden.

  Danach suchen Sie rechtwinklige Dreiecke.
\end{rezept}

\subsection{Umfang und Fläche}\index{Kreis!Umfang}

\begin{gesetz}{Kreislinie}{}\index{PI@$\pi$ (Pi)}\index{$\pi$}
  Die Länge der Kreislinie wird aus dem Durchmesser $d=2r$  mittels Kreiszahl
  $\pi$ berechnet. Es gilt für den Umfang $U$:
  $$U = 2r\pi = d\pi$$
\end{gesetz}

\begin{gesetz}{Kreisfläche}{}\index{Kreisfläche}
  Die Kreisfläche $A$ eines Kreises mit Radius $r$ wird wie folgt
  berechnet:
  $$A = r^2\pi$$
\end{gesetz}
Herleitung
\TNTeop{Pizza: Schneiden, auslegen = halbe Rechtecksfläche.}

%%%%%%%%%%%%%%%%%%%%%%%%%%%%%%%%%%5
\subsection{Kreisteile}
\begin{gesetz}{Kreisring}{}\index{Kreisring}
  Die Kreisringfläche ist die Differenz der umgebenden Kreisfläche
  (Radius $R$) und
  der inneren Kreisfläche (Radius $r$):

  $$A = A(R) - A(r) = R^2\pi - r^2\pi = (R^2-r^2)\pi$$
\end{gesetz}

\begin{gesetz}{Kreisbogen und
    Kreissektor}{}\index{Kreissektor}\index{Kreisbogen}
  Für den Sektorwinkel $\varphi$ werden Bogen $b$ und Sektorfläche
  $A_{SK}$ wie folgt berechnet:
  $$b = 2r\pi \cdot{}\frac{\varphi}{360\degre} =
  r\pi\cdot{}\frac{\varphi}{180\degre}$$
  $$A_{SK} = r^2\pi\cdot{}\frac{\varphi}{360\degre} = \frac12\cdot{}b\cdot{}r$$
  \end{gesetz}
  
\begin{rezept}{Geometrische Aufgaben}{}
Um geometrische Aufgaben zu lösen, hat bei mir folgendes meist geholfen:

\begin{enumerate}
\item Machen Sie eine Skizze
\item Machen Sie eine möglichst genaue Konstruktion
\item Geben Sie Gegebenem und Gesuchtem Namen
\item Verwenden Sie Farben für Gegebenes
\item Verwenden Sie die selben Farben (od. Symbole) für die selben Streckenlängen, Winkel, Flächen
\item Bei Aufgaben mit Kreisen: Verbinden Sie die Mittelpunkte 
\item Suchen Sie rechtwinklige Dreiecke (Pythagoras)

\end{enumerate}
\end{rezept}  

%Für das Kreissegment Siehe \cite{marthaler20geom} Seite 61 Kap. 4.2.3.
%
%\subsection*{Aufgaben}
%%% Kreisfläche
%%%\TALSAadBFWG{43ff (Kreisfläche)}{159. 160. 165. 168.}
%\AadBMTG{63}{1., 2., 3. und 4.}
%%% Kreissektor und Segment
%%%\TALSAadBFWG{48ff(Kreissektor)}{179. 180. 183. 185. 187. 188. 189. 192. 194.}
%\AadBMTG{65}{12., 13. (Bogenmaß), 14., 15., 17.}
%\AadBMTG{123}{39. a) b) c) d)}%% Aufgabe schon vorgelöst
%%% Kreissegment
%%%\TALSAadBFWG{51ff (Kreissegment)}{197. 198. 200. 204. 206.}
%
%
%%% Vermischte Aufgaben
%%%\TALSAadBFWG{53ff (vermischte Aufgaben)}{209. 215. 219. 223.}
\newpage
%\input{tals/plani/Strahlensaetze}\newpage
%\input{tals/plani/Aehnlichkeit}\newpage





%% Datenanalyse
%%
% Metapackage TALS Planimetrie
\part{Geometrie I}\index{Planimetrie|textbf}
\renewcommand{\bbwPartID}{PLANI}
\input{tals/trig1/Pythagoras}\newpage
%\input{tals/plani/Vielecke}
%%\input{tals/plani/Elementare_Objekte}\newpage
%%\input{tals/plani/Elemente_und_Zusammenhaenge}\newpage
%%
%% 2019 07 04 Ph. G. Freimann
%%

\section{Kreise}\index{Kreisberechnungen}\index{Berechnungen am Kreis}
\sectuntertitel{Warum sind Seeräuber so schlecht in Geometrie? --- Weil sie $\pi$ raten!}
%%%%%%%%%%%%%%%%%%%%%%%%%%%%%%%%%%%%%%%%%%%%%%%%%%%%%%%%%%%%%%%%%%%%%%%%%%%%%%%%%


\subsection*{Lernziele}

\begin{itemize}
\item einfache Kreisberechnungen
\item Kreisring
\item Tangente
\item Sehne (und Sekante)
\item Segment und Sektor
\end{itemize}

\TadBMTG{56}{4}

\newpage


\subsection{Kreisberührung}\index{Kreisberührung!Planimetrie}


\bbwCenterGraphic{8cm}{tals/plani/img/KleinsterKreis.png}
In obigem Kreis sind zwei kleinere Kreise einbeschrieben. Berechnen
Sie den Radius $k$ des kleinsten Kreises aus dem gegebenen Radius $r$ des
großen Kreises.
\TNT{12}{
  \bbwCenterGraphic{8cm}{tals/plani/img/KleinsterKreisLoesung.png}
  $$\Delta A: x^2 + k^2 = (r-k)^2$$
  $$\Delta B: x^2 + (\frac{r}2 - k)^2 = (\frac{r}2 + k)^2$$
  Ausmultiplizieren und 2. Gleichung von 1. Gleichung subtrahieren:
  $$k^2 - \frac{r^2}4 +rk - k^2  =r^2  -2rk - \frac{r^2}4 - rk$$\
  $$-\frac{r^2}4 + rk = r^2 - 3rk - \frac{r^2}4$$
  $$rk = r^2 - 3rk$$
  $$4k = r$$
  $$k = \frac{r}4$$
}%% END TNT
\newpage


\begin{rezept}{Kreisberührung}{}
  Bei Aufgaben, bei denen sich zwei Kreise \textbf{berühren}, ist es
  von Vorteil, die Mittelpunkte der Kreise mit den Tangentenpunkten zu
  verbinden.

  Danach suchen Sie rechtwinklige Dreiecke.
\end{rezept}

\subsection{Umfang und Fläche}\index{Kreis!Umfang}

\begin{gesetz}{Kreislinie}{}\index{PI@$\pi$ (Pi)}\index{$\pi$}
  Die Länge der Kreislinie wird aus dem Durchmesser $d=2r$  mittels Kreiszahl
  $\pi$ berechnet. Es gilt für den Umfang $U$:
  $$U = 2r\pi = d\pi$$
\end{gesetz}

\begin{gesetz}{Kreisfläche}{}\index{Kreisfläche}
  Die Kreisfläche $A$ eines Kreises mit Radius $r$ wird wie folgt
  berechnet:
  $$A = r^2\pi$$
\end{gesetz}
Herleitung
\TNTeop{Pizza: Schneiden, auslegen = halbe Rechtecksfläche.}

%%%%%%%%%%%%%%%%%%%%%%%%%%%%%%%%%%5
\subsection{Kreisteile}
\begin{gesetz}{Kreisring}{}\index{Kreisring}
  Die Kreisringfläche ist die Differenz der umgebenden Kreisfläche
  (Radius $R$) und
  der inneren Kreisfläche (Radius $r$):

  $$A = A(R) - A(r) = R^2\pi - r^2\pi = (R^2-r^2)\pi$$
\end{gesetz}

\begin{gesetz}{Kreisbogen und
    Kreissektor}{}\index{Kreissektor}\index{Kreisbogen}
  Für den Sektorwinkel $\varphi$ werden Bogen $b$ und Sektorfläche
  $A_{SK}$ wie folgt berechnet:
  $$b = 2r\pi \cdot{}\frac{\varphi}{360\degre} =
  r\pi\cdot{}\frac{\varphi}{180\degre}$$
  $$A_{SK} = r^2\pi\cdot{}\frac{\varphi}{360\degre} = \frac12\cdot{}b\cdot{}r$$
  \end{gesetz}
  
\begin{rezept}{Geometrische Aufgaben}{}
Um geometrische Aufgaben zu lösen, hat bei mir folgendes meist geholfen:

\begin{enumerate}
\item Machen Sie eine Skizze
\item Machen Sie eine möglichst genaue Konstruktion
\item Geben Sie Gegebenem und Gesuchtem Namen
\item Verwenden Sie Farben für Gegebenes
\item Verwenden Sie die selben Farben (od. Symbole) für die selben Streckenlängen, Winkel, Flächen
\item Bei Aufgaben mit Kreisen: Verbinden Sie die Mittelpunkte 
\item Suchen Sie rechtwinklige Dreiecke (Pythagoras)

\end{enumerate}
\end{rezept}  

%Für das Kreissegment Siehe \cite{marthaler20geom} Seite 61 Kap. 4.2.3.
%
%\subsection*{Aufgaben}
%%% Kreisfläche
%%%\TALSAadBFWG{43ff (Kreisfläche)}{159. 160. 165. 168.}
%\AadBMTG{63}{1., 2., 3. und 4.}
%%% Kreissektor und Segment
%%%\TALSAadBFWG{48ff(Kreissektor)}{179. 180. 183. 185. 187. 188. 189. 192. 194.}
%\AadBMTG{65}{12., 13. (Bogenmaß), 14., 15., 17.}
%\AadBMTG{123}{39. a) b) c) d)}%% Aufgabe schon vorgelöst
%%% Kreissegment
%%%\TALSAadBFWG{51ff (Kreissegment)}{197. 198. 200. 204. 206.}
%
%
%%% Vermischte Aufgaben
%%%\TALSAadBFWG{53ff (vermischte Aufgaben)}{209. 215. 219. 223.}
\newpage
%\input{tals/plani/Strahlensaetze}\newpage
%\input{tals/plani/Aehnlichkeit}\newpage





%% Funktionen II TALS
%%
% Metapackage TALS Planimetrie
\part{Geometrie I}\index{Planimetrie|textbf}
\renewcommand{\bbwPartID}{PLANI}
\input{tals/trig1/Pythagoras}\newpage
%\input{tals/plani/Vielecke}
%%\input{tals/plani/Elementare_Objekte}\newpage
%%\input{tals/plani/Elemente_und_Zusammenhaenge}\newpage
%%
%% 2019 07 04 Ph. G. Freimann
%%

\section{Kreise}\index{Kreisberechnungen}\index{Berechnungen am Kreis}
\sectuntertitel{Warum sind Seeräuber so schlecht in Geometrie? --- Weil sie $\pi$ raten!}
%%%%%%%%%%%%%%%%%%%%%%%%%%%%%%%%%%%%%%%%%%%%%%%%%%%%%%%%%%%%%%%%%%%%%%%%%%%%%%%%%


\subsection*{Lernziele}

\begin{itemize}
\item einfache Kreisberechnungen
\item Kreisring
\item Tangente
\item Sehne (und Sekante)
\item Segment und Sektor
\end{itemize}

\TadBMTG{56}{4}

\newpage


\subsection{Kreisberührung}\index{Kreisberührung!Planimetrie}


\bbwCenterGraphic{8cm}{tals/plani/img/KleinsterKreis.png}
In obigem Kreis sind zwei kleinere Kreise einbeschrieben. Berechnen
Sie den Radius $k$ des kleinsten Kreises aus dem gegebenen Radius $r$ des
großen Kreises.
\TNT{12}{
  \bbwCenterGraphic{8cm}{tals/plani/img/KleinsterKreisLoesung.png}
  $$\Delta A: x^2 + k^2 = (r-k)^2$$
  $$\Delta B: x^2 + (\frac{r}2 - k)^2 = (\frac{r}2 + k)^2$$
  Ausmultiplizieren und 2. Gleichung von 1. Gleichung subtrahieren:
  $$k^2 - \frac{r^2}4 +rk - k^2  =r^2  -2rk - \frac{r^2}4 - rk$$\
  $$-\frac{r^2}4 + rk = r^2 - 3rk - \frac{r^2}4$$
  $$rk = r^2 - 3rk$$
  $$4k = r$$
  $$k = \frac{r}4$$
}%% END TNT
\newpage


\begin{rezept}{Kreisberührung}{}
  Bei Aufgaben, bei denen sich zwei Kreise \textbf{berühren}, ist es
  von Vorteil, die Mittelpunkte der Kreise mit den Tangentenpunkten zu
  verbinden.

  Danach suchen Sie rechtwinklige Dreiecke.
\end{rezept}

\subsection{Umfang und Fläche}\index{Kreis!Umfang}

\begin{gesetz}{Kreislinie}{}\index{PI@$\pi$ (Pi)}\index{$\pi$}
  Die Länge der Kreislinie wird aus dem Durchmesser $d=2r$  mittels Kreiszahl
  $\pi$ berechnet. Es gilt für den Umfang $U$:
  $$U = 2r\pi = d\pi$$
\end{gesetz}

\begin{gesetz}{Kreisfläche}{}\index{Kreisfläche}
  Die Kreisfläche $A$ eines Kreises mit Radius $r$ wird wie folgt
  berechnet:
  $$A = r^2\pi$$
\end{gesetz}
Herleitung
\TNTeop{Pizza: Schneiden, auslegen = halbe Rechtecksfläche.}

%%%%%%%%%%%%%%%%%%%%%%%%%%%%%%%%%%5
\subsection{Kreisteile}
\begin{gesetz}{Kreisring}{}\index{Kreisring}
  Die Kreisringfläche ist die Differenz der umgebenden Kreisfläche
  (Radius $R$) und
  der inneren Kreisfläche (Radius $r$):

  $$A = A(R) - A(r) = R^2\pi - r^2\pi = (R^2-r^2)\pi$$
\end{gesetz}

\begin{gesetz}{Kreisbogen und
    Kreissektor}{}\index{Kreissektor}\index{Kreisbogen}
  Für den Sektorwinkel $\varphi$ werden Bogen $b$ und Sektorfläche
  $A_{SK}$ wie folgt berechnet:
  $$b = 2r\pi \cdot{}\frac{\varphi}{360\degre} =
  r\pi\cdot{}\frac{\varphi}{180\degre}$$
  $$A_{SK} = r^2\pi\cdot{}\frac{\varphi}{360\degre} = \frac12\cdot{}b\cdot{}r$$
  \end{gesetz}
  
\begin{rezept}{Geometrische Aufgaben}{}
Um geometrische Aufgaben zu lösen, hat bei mir folgendes meist geholfen:

\begin{enumerate}
\item Machen Sie eine Skizze
\item Machen Sie eine möglichst genaue Konstruktion
\item Geben Sie Gegebenem und Gesuchtem Namen
\item Verwenden Sie Farben für Gegebenes
\item Verwenden Sie die selben Farben (od. Symbole) für die selben Streckenlängen, Winkel, Flächen
\item Bei Aufgaben mit Kreisen: Verbinden Sie die Mittelpunkte 
\item Suchen Sie rechtwinklige Dreiecke (Pythagoras)

\end{enumerate}
\end{rezept}  

%Für das Kreissegment Siehe \cite{marthaler20geom} Seite 61 Kap. 4.2.3.
%
%\subsection*{Aufgaben}
%%% Kreisfläche
%%%\TALSAadBFWG{43ff (Kreisfläche)}{159. 160. 165. 168.}
%\AadBMTG{63}{1., 2., 3. und 4.}
%%% Kreissektor und Segment
%%%\TALSAadBFWG{48ff(Kreissektor)}{179. 180. 183. 185. 187. 188. 189. 192. 194.}
%\AadBMTG{65}{12., 13. (Bogenmaß), 14., 15., 17.}
%\AadBMTG{123}{39. a) b) c) d)}%% Aufgabe schon vorgelöst
%%% Kreissegment
%%%\TALSAadBFWG{51ff (Kreissegment)}{197. 198. 200. 204. 206.}
%
%
%%% Vermischte Aufgaben
%%%\TALSAadBFWG{53ff (vermischte Aufgaben)}{209. 215. 219. 223.}
\newpage
%\input{tals/plani/Strahlensaetze}\newpage
%\input{tals/plani/Aehnlichkeit}\newpage





%% Trigonometrie III
%%
% Metapackage TALS Planimetrie
\part{Geometrie I}\index{Planimetrie|textbf}
\renewcommand{\bbwPartID}{PLANI}
\input{tals/trig1/Pythagoras}\newpage
%\input{tals/plani/Vielecke}
%%\input{tals/plani/Elementare_Objekte}\newpage
%%\input{tals/plani/Elemente_und_Zusammenhaenge}\newpage
%%
%% 2019 07 04 Ph. G. Freimann
%%

\section{Kreise}\index{Kreisberechnungen}\index{Berechnungen am Kreis}
\sectuntertitel{Warum sind Seeräuber so schlecht in Geometrie? --- Weil sie $\pi$ raten!}
%%%%%%%%%%%%%%%%%%%%%%%%%%%%%%%%%%%%%%%%%%%%%%%%%%%%%%%%%%%%%%%%%%%%%%%%%%%%%%%%%


\subsection*{Lernziele}

\begin{itemize}
\item einfache Kreisberechnungen
\item Kreisring
\item Tangente
\item Sehne (und Sekante)
\item Segment und Sektor
\end{itemize}

\TadBMTG{56}{4}

\newpage


\subsection{Kreisberührung}\index{Kreisberührung!Planimetrie}


\bbwCenterGraphic{8cm}{tals/plani/img/KleinsterKreis.png}
In obigem Kreis sind zwei kleinere Kreise einbeschrieben. Berechnen
Sie den Radius $k$ des kleinsten Kreises aus dem gegebenen Radius $r$ des
großen Kreises.
\TNT{12}{
  \bbwCenterGraphic{8cm}{tals/plani/img/KleinsterKreisLoesung.png}
  $$\Delta A: x^2 + k^2 = (r-k)^2$$
  $$\Delta B: x^2 + (\frac{r}2 - k)^2 = (\frac{r}2 + k)^2$$
  Ausmultiplizieren und 2. Gleichung von 1. Gleichung subtrahieren:
  $$k^2 - \frac{r^2}4 +rk - k^2  =r^2  -2rk - \frac{r^2}4 - rk$$\
  $$-\frac{r^2}4 + rk = r^2 - 3rk - \frac{r^2}4$$
  $$rk = r^2 - 3rk$$
  $$4k = r$$
  $$k = \frac{r}4$$
}%% END TNT
\newpage


\begin{rezept}{Kreisberührung}{}
  Bei Aufgaben, bei denen sich zwei Kreise \textbf{berühren}, ist es
  von Vorteil, die Mittelpunkte der Kreise mit den Tangentenpunkten zu
  verbinden.

  Danach suchen Sie rechtwinklige Dreiecke.
\end{rezept}

\subsection{Umfang und Fläche}\index{Kreis!Umfang}

\begin{gesetz}{Kreislinie}{}\index{PI@$\pi$ (Pi)}\index{$\pi$}
  Die Länge der Kreislinie wird aus dem Durchmesser $d=2r$  mittels Kreiszahl
  $\pi$ berechnet. Es gilt für den Umfang $U$:
  $$U = 2r\pi = d\pi$$
\end{gesetz}

\begin{gesetz}{Kreisfläche}{}\index{Kreisfläche}
  Die Kreisfläche $A$ eines Kreises mit Radius $r$ wird wie folgt
  berechnet:
  $$A = r^2\pi$$
\end{gesetz}
Herleitung
\TNTeop{Pizza: Schneiden, auslegen = halbe Rechtecksfläche.}

%%%%%%%%%%%%%%%%%%%%%%%%%%%%%%%%%%5
\subsection{Kreisteile}
\begin{gesetz}{Kreisring}{}\index{Kreisring}
  Die Kreisringfläche ist die Differenz der umgebenden Kreisfläche
  (Radius $R$) und
  der inneren Kreisfläche (Radius $r$):

  $$A = A(R) - A(r) = R^2\pi - r^2\pi = (R^2-r^2)\pi$$
\end{gesetz}

\begin{gesetz}{Kreisbogen und
    Kreissektor}{}\index{Kreissektor}\index{Kreisbogen}
  Für den Sektorwinkel $\varphi$ werden Bogen $b$ und Sektorfläche
  $A_{SK}$ wie folgt berechnet:
  $$b = 2r\pi \cdot{}\frac{\varphi}{360\degre} =
  r\pi\cdot{}\frac{\varphi}{180\degre}$$
  $$A_{SK} = r^2\pi\cdot{}\frac{\varphi}{360\degre} = \frac12\cdot{}b\cdot{}r$$
  \end{gesetz}
  
\begin{rezept}{Geometrische Aufgaben}{}
Um geometrische Aufgaben zu lösen, hat bei mir folgendes meist geholfen:

\begin{enumerate}
\item Machen Sie eine Skizze
\item Machen Sie eine möglichst genaue Konstruktion
\item Geben Sie Gegebenem und Gesuchtem Namen
\item Verwenden Sie Farben für Gegebenes
\item Verwenden Sie die selben Farben (od. Symbole) für die selben Streckenlängen, Winkel, Flächen
\item Bei Aufgaben mit Kreisen: Verbinden Sie die Mittelpunkte 
\item Suchen Sie rechtwinklige Dreiecke (Pythagoras)

\end{enumerate}
\end{rezept}  

%Für das Kreissegment Siehe \cite{marthaler20geom} Seite 61 Kap. 4.2.3.
%
%\subsection*{Aufgaben}
%%% Kreisfläche
%%%\TALSAadBFWG{43ff (Kreisfläche)}{159. 160. 165. 168.}
%\AadBMTG{63}{1., 2., 3. und 4.}
%%% Kreissektor und Segment
%%%\TALSAadBFWG{48ff(Kreissektor)}{179. 180. 183. 185. 187. 188. 189. 192. 194.}
%\AadBMTG{65}{12., 13. (Bogenmaß), 14., 15., 17.}
%\AadBMTG{123}{39. a) b) c) d)}%% Aufgabe schon vorgelöst
%%% Kreissegment
%%%\TALSAadBFWG{51ff (Kreissegment)}{197. 198. 200. 204. 206.}
%
%
%%% Vermischte Aufgaben
%%%\TALSAadBFWG{53ff (vermischte Aufgaben)}{209. 215. 219. 223.}
\newpage
%\input{tals/plani/Strahlensaetze}\newpage
%\input{tals/plani/Aehnlichkeit}\newpage





%% Trigonometrie I%
%%
% Metapackage TALS Planimetrie
\part{Geometrie I}\index{Planimetrie|textbf}
\renewcommand{\bbwPartID}{PLANI}
\input{tals/trig1/Pythagoras}\newpage
%\input{tals/plani/Vielecke}
%%\input{tals/plani/Elementare_Objekte}\newpage
%%\input{tals/plani/Elemente_und_Zusammenhaenge}\newpage
%%
%% 2019 07 04 Ph. G. Freimann
%%

\section{Kreise}\index{Kreisberechnungen}\index{Berechnungen am Kreis}
\sectuntertitel{Warum sind Seeräuber so schlecht in Geometrie? --- Weil sie $\pi$ raten!}
%%%%%%%%%%%%%%%%%%%%%%%%%%%%%%%%%%%%%%%%%%%%%%%%%%%%%%%%%%%%%%%%%%%%%%%%%%%%%%%%%


\subsection*{Lernziele}

\begin{itemize}
\item einfache Kreisberechnungen
\item Kreisring
\item Tangente
\item Sehne (und Sekante)
\item Segment und Sektor
\end{itemize}

\TadBMTG{56}{4}

\newpage


\subsection{Kreisberührung}\index{Kreisberührung!Planimetrie}


\bbwCenterGraphic{8cm}{tals/plani/img/KleinsterKreis.png}
In obigem Kreis sind zwei kleinere Kreise einbeschrieben. Berechnen
Sie den Radius $k$ des kleinsten Kreises aus dem gegebenen Radius $r$ des
großen Kreises.
\TNT{12}{
  \bbwCenterGraphic{8cm}{tals/plani/img/KleinsterKreisLoesung.png}
  $$\Delta A: x^2 + k^2 = (r-k)^2$$
  $$\Delta B: x^2 + (\frac{r}2 - k)^2 = (\frac{r}2 + k)^2$$
  Ausmultiplizieren und 2. Gleichung von 1. Gleichung subtrahieren:
  $$k^2 - \frac{r^2}4 +rk - k^2  =r^2  -2rk - \frac{r^2}4 - rk$$\
  $$-\frac{r^2}4 + rk = r^2 - 3rk - \frac{r^2}4$$
  $$rk = r^2 - 3rk$$
  $$4k = r$$
  $$k = \frac{r}4$$
}%% END TNT
\newpage


\begin{rezept}{Kreisberührung}{}
  Bei Aufgaben, bei denen sich zwei Kreise \textbf{berühren}, ist es
  von Vorteil, die Mittelpunkte der Kreise mit den Tangentenpunkten zu
  verbinden.

  Danach suchen Sie rechtwinklige Dreiecke.
\end{rezept}

\subsection{Umfang und Fläche}\index{Kreis!Umfang}

\begin{gesetz}{Kreislinie}{}\index{PI@$\pi$ (Pi)}\index{$\pi$}
  Die Länge der Kreislinie wird aus dem Durchmesser $d=2r$  mittels Kreiszahl
  $\pi$ berechnet. Es gilt für den Umfang $U$:
  $$U = 2r\pi = d\pi$$
\end{gesetz}

\begin{gesetz}{Kreisfläche}{}\index{Kreisfläche}
  Die Kreisfläche $A$ eines Kreises mit Radius $r$ wird wie folgt
  berechnet:
  $$A = r^2\pi$$
\end{gesetz}
Herleitung
\TNTeop{Pizza: Schneiden, auslegen = halbe Rechtecksfläche.}

%%%%%%%%%%%%%%%%%%%%%%%%%%%%%%%%%%5
\subsection{Kreisteile}
\begin{gesetz}{Kreisring}{}\index{Kreisring}
  Die Kreisringfläche ist die Differenz der umgebenden Kreisfläche
  (Radius $R$) und
  der inneren Kreisfläche (Radius $r$):

  $$A = A(R) - A(r) = R^2\pi - r^2\pi = (R^2-r^2)\pi$$
\end{gesetz}

\begin{gesetz}{Kreisbogen und
    Kreissektor}{}\index{Kreissektor}\index{Kreisbogen}
  Für den Sektorwinkel $\varphi$ werden Bogen $b$ und Sektorfläche
  $A_{SK}$ wie folgt berechnet:
  $$b = 2r\pi \cdot{}\frac{\varphi}{360\degre} =
  r\pi\cdot{}\frac{\varphi}{180\degre}$$
  $$A_{SK} = r^2\pi\cdot{}\frac{\varphi}{360\degre} = \frac12\cdot{}b\cdot{}r$$
  \end{gesetz}
  
\begin{rezept}{Geometrische Aufgaben}{}
Um geometrische Aufgaben zu lösen, hat bei mir folgendes meist geholfen:

\begin{enumerate}
\item Machen Sie eine Skizze
\item Machen Sie eine möglichst genaue Konstruktion
\item Geben Sie Gegebenem und Gesuchtem Namen
\item Verwenden Sie Farben für Gegebenes
\item Verwenden Sie die selben Farben (od. Symbole) für die selben Streckenlängen, Winkel, Flächen
\item Bei Aufgaben mit Kreisen: Verbinden Sie die Mittelpunkte 
\item Suchen Sie rechtwinklige Dreiecke (Pythagoras)

\end{enumerate}
\end{rezept}  

%Für das Kreissegment Siehe \cite{marthaler20geom} Seite 61 Kap. 4.2.3.
%
%\subsection*{Aufgaben}
%%% Kreisfläche
%%%\TALSAadBFWG{43ff (Kreisfläche)}{159. 160. 165. 168.}
%\AadBMTG{63}{1., 2., 3. und 4.}
%%% Kreissektor und Segment
%%%\TALSAadBFWG{48ff(Kreissektor)}{179. 180. 183. 185. 187. 188. 189. 192. 194.}
%\AadBMTG{65}{12., 13. (Bogenmaß), 14., 15., 17.}
%\AadBMTG{123}{39. a) b) c) d)}%% Aufgabe schon vorgelöst
%%% Kreissegment
%%%\TALSAadBFWG{51ff (Kreissegment)}{197. 198. 200. 204. 206.}
%
%
%%% Vermischte Aufgaben
%%%\TALSAadBFWG{53ff (vermischte Aufgaben)}{209. 215. 219. 223.}
\newpage
%\input{tals/plani/Strahlensaetze}\newpage
%\input{tals/plani/Aehnlichkeit}\newpage





%% Trigonometrie I%
%%
% Metapackage TALS Planimetrie
\part{Geometrie I}\index{Planimetrie|textbf}
\renewcommand{\bbwPartID}{PLANI}
\input{tals/trig1/Pythagoras}\newpage
%\input{tals/plani/Vielecke}
%%\input{tals/plani/Elementare_Objekte}\newpage
%%\input{tals/plani/Elemente_und_Zusammenhaenge}\newpage
%%
%% 2019 07 04 Ph. G. Freimann
%%

\section{Kreise}\index{Kreisberechnungen}\index{Berechnungen am Kreis}
\sectuntertitel{Warum sind Seeräuber so schlecht in Geometrie? --- Weil sie $\pi$ raten!}
%%%%%%%%%%%%%%%%%%%%%%%%%%%%%%%%%%%%%%%%%%%%%%%%%%%%%%%%%%%%%%%%%%%%%%%%%%%%%%%%%


\subsection*{Lernziele}

\begin{itemize}
\item einfache Kreisberechnungen
\item Kreisring
\item Tangente
\item Sehne (und Sekante)
\item Segment und Sektor
\end{itemize}

\TadBMTG{56}{4}

\newpage


\subsection{Kreisberührung}\index{Kreisberührung!Planimetrie}


\bbwCenterGraphic{8cm}{tals/plani/img/KleinsterKreis.png}
In obigem Kreis sind zwei kleinere Kreise einbeschrieben. Berechnen
Sie den Radius $k$ des kleinsten Kreises aus dem gegebenen Radius $r$ des
großen Kreises.
\TNT{12}{
  \bbwCenterGraphic{8cm}{tals/plani/img/KleinsterKreisLoesung.png}
  $$\Delta A: x^2 + k^2 = (r-k)^2$$
  $$\Delta B: x^2 + (\frac{r}2 - k)^2 = (\frac{r}2 + k)^2$$
  Ausmultiplizieren und 2. Gleichung von 1. Gleichung subtrahieren:
  $$k^2 - \frac{r^2}4 +rk - k^2  =r^2  -2rk - \frac{r^2}4 - rk$$\
  $$-\frac{r^2}4 + rk = r^2 - 3rk - \frac{r^2}4$$
  $$rk = r^2 - 3rk$$
  $$4k = r$$
  $$k = \frac{r}4$$
}%% END TNT
\newpage


\begin{rezept}{Kreisberührung}{}
  Bei Aufgaben, bei denen sich zwei Kreise \textbf{berühren}, ist es
  von Vorteil, die Mittelpunkte der Kreise mit den Tangentenpunkten zu
  verbinden.

  Danach suchen Sie rechtwinklige Dreiecke.
\end{rezept}

\subsection{Umfang und Fläche}\index{Kreis!Umfang}

\begin{gesetz}{Kreislinie}{}\index{PI@$\pi$ (Pi)}\index{$\pi$}
  Die Länge der Kreislinie wird aus dem Durchmesser $d=2r$  mittels Kreiszahl
  $\pi$ berechnet. Es gilt für den Umfang $U$:
  $$U = 2r\pi = d\pi$$
\end{gesetz}

\begin{gesetz}{Kreisfläche}{}\index{Kreisfläche}
  Die Kreisfläche $A$ eines Kreises mit Radius $r$ wird wie folgt
  berechnet:
  $$A = r^2\pi$$
\end{gesetz}
Herleitung
\TNTeop{Pizza: Schneiden, auslegen = halbe Rechtecksfläche.}

%%%%%%%%%%%%%%%%%%%%%%%%%%%%%%%%%%5
\subsection{Kreisteile}
\begin{gesetz}{Kreisring}{}\index{Kreisring}
  Die Kreisringfläche ist die Differenz der umgebenden Kreisfläche
  (Radius $R$) und
  der inneren Kreisfläche (Radius $r$):

  $$A = A(R) - A(r) = R^2\pi - r^2\pi = (R^2-r^2)\pi$$
\end{gesetz}

\begin{gesetz}{Kreisbogen und
    Kreissektor}{}\index{Kreissektor}\index{Kreisbogen}
  Für den Sektorwinkel $\varphi$ werden Bogen $b$ und Sektorfläche
  $A_{SK}$ wie folgt berechnet:
  $$b = 2r\pi \cdot{}\frac{\varphi}{360\degre} =
  r\pi\cdot{}\frac{\varphi}{180\degre}$$
  $$A_{SK} = r^2\pi\cdot{}\frac{\varphi}{360\degre} = \frac12\cdot{}b\cdot{}r$$
  \end{gesetz}
  
\begin{rezept}{Geometrische Aufgaben}{}
Um geometrische Aufgaben zu lösen, hat bei mir folgendes meist geholfen:

\begin{enumerate}
\item Machen Sie eine Skizze
\item Machen Sie eine möglichst genaue Konstruktion
\item Geben Sie Gegebenem und Gesuchtem Namen
\item Verwenden Sie Farben für Gegebenes
\item Verwenden Sie die selben Farben (od. Symbole) für die selben Streckenlängen, Winkel, Flächen
\item Bei Aufgaben mit Kreisen: Verbinden Sie die Mittelpunkte 
\item Suchen Sie rechtwinklige Dreiecke (Pythagoras)

\end{enumerate}
\end{rezept}  

%Für das Kreissegment Siehe \cite{marthaler20geom} Seite 61 Kap. 4.2.3.
%
%\subsection*{Aufgaben}
%%% Kreisfläche
%%%\TALSAadBFWG{43ff (Kreisfläche)}{159. 160. 165. 168.}
%\AadBMTG{63}{1., 2., 3. und 4.}
%%% Kreissektor und Segment
%%%\TALSAadBFWG{48ff(Kreissektor)}{179. 180. 183. 185. 187. 188. 189. 192. 194.}
%\AadBMTG{65}{12., 13. (Bogenmaß), 14., 15., 17.}
%\AadBMTG{123}{39. a) b) c) d)}%% Aufgabe schon vorgelöst
%%% Kreissegment
%%%\TALSAadBFWG{51ff (Kreissegment)}{197. 198. 200. 204. 206.}
%
%
%%% Vermischte Aufgaben
%%%\TALSAadBFWG{53ff (vermischte Aufgaben)}{209. 215. 219. 223.}
\newpage
%\input{tals/plani/Strahlensaetze}\newpage
%\input{tals/plani/Aehnlichkeit}\newpage





\scriptEnde{}