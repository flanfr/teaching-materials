\subsection{Lineare Abhängigkeit}\index{abhängig!linear}\index{Lineare
  Abhängigkeit}

Lösen Sie das folgende Gleichungssystem vorerst mit dem
Taschenrechner:

\gleichungZZ{9x-6y}{18}{30x-20y}{60}

Die Lösung ist nicht vielsagend.

Durch die Additionsmethode erhalten wir folgendes Gleichungssystem:

\gleichungZZ{90x-60y}{180}{90x-60y}{180}
oder $$0=0$$.

Dies bedeutet, wir haben durch Äquivalenzumformungen nun zweimal die selbe
Gleichung da stehen. Wir können $x$ nur in Abhängigkeit von $y$
berechnen (mehr nicht):

$$x=\frac{6+2y}{3}$$

Oder wir können $y$ in Abhängigkeit von $x$ berechnen:
$$y=\frac{3x-6}{2}$$

Wichtig ist vor allem, dass Sie auch die Antwort des Taschenrechners verstehen!
