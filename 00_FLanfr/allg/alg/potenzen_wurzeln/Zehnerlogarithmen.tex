%%
%% 2019 07 11 Ph. G. Freimann
%%

\section{Zehnerlogarithmus}\index{Logarithmus!Basis 10}\index{Zehnerlogarithmus}
\sectuntertitel{\textit{logos}: Das Verhältnis, \textit{arithmos}: Die Zahl}

\GESOTadBMTA{97}{6.1}

%%%%%%%%%%%%%%%%%%%%%%%%%%%%%%%%%%%%%%%%%%%%%%%%%%%%%%%%%%%%%%%%%%%%%%%%%%%%%%%%%
\subsection*{Lernziele}

\begin{itemize}
\item Zehnerpotenzen
\item wissenschaftliche Notation (Repetition)
\item Zehnerlogarithmus
\item Notation
\item Logarithmengesetze I
\item Beispiele: Richterskala (Erdbeben), dB (Lautstärke), pH («Säuregehalt»)
\end{itemize}

\ifisALLINONE{
\subsubsection*{wissenschaftliche Notation}
Die wissenschaftliche Notation wurde bereits
eingeführt. Zur Erinnerung:\totalref{wissenschaftlicheNotation}.
}\fi{}%% END ALL IN ONE
\newpage

\subsection{Definition Zehnerlogarithmus}\index{Logarithmus!Definition Zehnerlogarithmus}

\renewcommand{\arraystretch}{2}
\begin{tabular}{cc|cc}
  \hline
  Potenz               & ${\color{blue}10}^{\color{red}3} = {\color{ForestGreen}x}$     & ${\color{ForestGreen}x}=\LoesungsRaumLang{{\color{blue}10}\cdot{}{\color{blue}10}\cdot{}{\color{blue}10} = {\color{ForestGreen}1000}}$ & \LoesungsRaum{Potenzwert} \\\hline
  Potenzgleichung      & ${\color{blue}x}^{\color{red}3}  = {\color{ForestGreen}1000}$  & ${\color{blue}x}=\LoesungsRaumLang{\sqrt[{\color{red}3}]{{\color{ForestGreen}1000}}       = {\color{blue}10}}$   & \LoesungsRaum{3. Wurzel}  \\\hline
  Exponentialgleichung & ${\color{blue}10}^{\color{red}x} = {\color{ForestGreen}1000}$  & ${\color{red}x}=\LoesungsRaumLang{\lg(1000) = {\color{red}3}}$                                  & \LoesungsRaum{Logarithmus} \\\hline
  \end{tabular} 
\renewcommand{\arraystretch}{1}

\vspace{5mm}

Allgemein für ${\color{red}z}\in \mathbb{Z}$:

\begin{gesetz}{Logarithmus = Exponent}{}
  $$\lg({\color{green} 10^z}) = {\color{red} z}$$
\end{gesetz}


Definition für $x\in\mathbb{R}, p \in \mathbb{R}^{+}\backslash\{0\}$:
\begin{definition}{Logarithmus zur Basis 10}{}
  \begin{center}
    ${\color{blue}10}^{\color{red}x}={\color{ForestGreen}p}$
    $\Longleftrightarrow$
    ${\color{red}x} = {\color{blue}\lg}({\color{ForestGreen}p})$
    \end{center}
\end{definition}

\begin{bemerkung}{}{}\index{Logarithmus!Definition}
  Der \textbf{Logarithmus} (hier $\color{red}x$) eines Potenzwertes
  (hier $\color{ForestGreen}p$) ist der Exponent ($\color{red}x$) in der
  Potenzschreibweise (${\color{blue}10}^{\color{red}x}$); und somit die \textbf{Umkehrung des Potenzierens}.
\end{bemerkung}

\textbf{Beispiele}:\\
\leserluft{}

$\lg(100 \cdot{} 1000) = \LoesungsRaum{5}$
\leserluft{}

$\lg(10^{-8}) = \LoesungsRaum{-8}$
\leserluft{}

$\lg(0.0001) = \LoesungsRaum{-4}$

$\lg(\frac1{100}) =\LoesungsRaumLang{\lg(10^{-2}) } = \LoesungsRaum{-2}$
\leserluft{}

$\lg(\frac1{1000}) =\LoesungsRaumLang{\lg(10^{-3}) } = \LoesungsRaum{-3}$

\newpage
\subsubsection{Logarithmus der Wurzel}
Wir erinnern uns: $$\sqrt{10} =\LoesungsRaum{ 10^\frac12}$$
\begin{beispiel}{Logarithmus der
    Wurzel}{beispiel_logarithmus_der_wurzel}

  Daraus folgt direkt $\lg(\sqrt{10}) = \LoesungsRaum{\lg(10^\frac12)}  = \LoesungsRaum{0.5}$.
\end{beispiel}

\subsubsection{Notation}
Der Zehnerlogarithmus $\lg()$ wird oft auch explizit mit der Basis 10
angegeben; dann wird $\log_{10}(\,\,)$ anstelle von $\lg(\,\,)$ geschrieben:

\begin{center}
\fbox{$\lg = \log_{10}$ }
\end{center}

\begin{bemerkung}{Taschenrechner}{}
  Auf vielen Taschenrechnern steht $\log$ anstatt $\lg$.

  \GESO{  \tiprobutton{ln_log}}
  \TALS{  \nspirebutton{10xlog}}
  \end{bemerkung}


\textbf{Beispiele}:\\


$\sqrt{10} = 10^{0.5} \approx \LoesungsRaum{3.16227766}$
\leserluft{}

$10^{2.5} \approx \LoesungsRaum{316.227766}$\\
\TRAINER{, denn $10^{2.5} = 10^{2+0.5} = 10^2 \cdot{} 10^{0.5} = 100\cdot{}\sqrt{10}$.}
\leserluft{}

$\lg(10^{2.5}) = \LoesungsRaum{2.5}$

\leserluft{}

$\lg(316.23) \approx \LoesungsRaum{2.500}$



\newpage

\subsection{Rechengesetze}
Per Definition gilt:

\begin{gesetz}{}{}
$$\lg(10) = 1$$
\end{gesetz}

\begin{gesetz}{}{}
$$\lg(1) = 0$$
\end{gesetz}

\begin{gesetz}{}{}
  $$10^{\lg(p)} = p$$
\end{gesetz}

Begründung:
\TNT{2.4}{ Durch beidseitiges Exponenzieren der Definition...

$$\lg(10^x) = x \Longrightarrow 10^{\lg(10^x)} = 10^x
\stackrel{p:=10^x}{\Longrightarrow} 10^{\lg(p)} = p$$
}%% END TNT

\begin{bemerkung}{negative Exponenten}{}
  Weil $10^x$ für alle $x$ immer positiv ausfällt, so gibt es keine
  Logarithmen von negativen Zahlen:

  $$\lg(a) \textrm{ ist definiert für } a\in\mathbb{R}^+\backslash\{0\}$$
  \end{bemerkung}


\newpage
\subsubsection{Multiplikatiosgesetz\GESO{ (optional)}}
\begin{gesetz}{}{}
  $\lg(r\cdot s) = \lg(r) + \lg(s)$
\end{gesetz}

Begründung:
\TNT{6.0}{
$10^n \cdot{} 10^m = 10^{n+m}.$
\\
Zahlenbeispiel:
$$\log(10^2\cdot{}10^4) = \log(10^{2+4}) = 2+4
= \log(10^2) + \log(10^4) $$

Oder allgemein (für $r = 10^n$ und $s = 10^m$)\footnote{\TRAINER{Wobei hier $n$ und $m$ nicht notwendigerweise in $\mathbb{N}$}}:
$$\lg(r \cdot s) = \lg(10^n \cdot 10^m) = \lg(10^{(n+m)}) = n+m = \lg(10^n) + \lg(10^m) = \lg(r) + \lg(s).$$
\vspace{15mm}
}%% END TNT

Beispiel: \TRAINER{Beim Vorzeigen mit den beiden 8ern von links und
  rechts beginnen.}

\TNT{2}{$${\color{gray}8=\lg(10^8)=\lg(10^{3+5})}=\lg(10^3\cdot{}10^5) =
\lg(10^3)+\lg(10^5){\color{gray}=3+5=8}$$}%% END TNT

\TALS{
  \textbf{Multiplizieren durch Addieren}\\

  Nach dem ebene gezeigten Gesetz gilt:
  \TNT{1.2}{  $$\lg(a\cdot{}b) = \lg(a) + \lg(b)$$}
  
  Daraus folgt:
\TNT{1.2}{  $$a\cdot{}b = 10^{\lg(a\cdot{}b)} = 10^{\lg(a) +
    \lg(b)}$$}

  Und somit können wir zwei Zahlen multiplizieren, indem wir ihre
  Logarithmen addieren:

  \aufgabenFarbe{Lösen Sie die Multiplikation auf dem Blatt im OLAT,
    indem Sie die Logarithmen der Zahlen addieren.}
}
\newpage


\subsection*{Aufgaben}
\GESOAadBMTA{102ff}{2. a) c) d) g) h) i), 3. a) c) e), 4. a), 12. a) c)
  und 18. a) b) c) e)} 
\GESOAadBMTA{108}{38.}
\GESOAadBMTA{107ff}{Optional: 36., 37. und 39.}

\TALSAadBMTA{60ff}{172. a) b) d), 173. f) h) k), 174. a) b) c), 175. b) c),
  176. a) b) f) und 179. a) f) h)}
\newpage

