%%
%% 2019 07 04 Ph. G. Freimann
%%

\section{Potenzen}\index{Potenzen}
\sectuntertitel{Ein dreifaches Hoch auf die Basen}

\GESOTadBMTA{57-59}{4.1, 4.2, 4.3 und 4.4}

%%%%%%%%%%%%%%%%%%%%%%%%%%%%%%%%%%%%%%%%%%%%%%%%%%%%%%%%%%%%%%%%%%%%%%%%%%%%%%%%%
\subsection*{Lernziele}

\begin{itemize}
\item Namen der Zehnerpotenzen
\item Definition: Exponent, Basis, Potenz
\item Rechengesetze
\item negative Exponenten
\item Hoch 0
\item Potenzen von Bruchtermen
\end{itemize}
\TALS{(\cite{frommenwiler17alg} S. 32 (Kap. 1.5))}
\newpage
%%\subsection{Definition}
%%$$a^n = \underbrace{a\cdot a \cdot a \cdot ... \cdot a}_{\textrm{n Faktoren}}$$

\subsection{Zehnerpotenzen}\index{Zehnerpotenzen}
\TALSTadBFWA{40}{1.5.5}

\bbwCenterGraphic{8cm}{allg/alg/potenzen_wurzeln/img/one_in_a_mellon.jpg}

Neben den im Buch (\GESO{\cite{marthaler21alg}
  S. 62}\TALS{\cite{frommenwiler17alg} S. 40}) angegebenen SI-Einheiten (Kilo, Mega, ...) sind die
Namen der positiven Zehnerpotenzen im Englischen anders als im
Deutschen. Hier zur Vollstänigkeit:

\begin{tabular}{lrclll}
Potenz    & Zahl & SI-Kürzel & SI-Vorsätze & Deutsch & Englisch\\
\hline\\
$10^{2}$   & 100  & h         & Hekto       & Hundert   & hundred\\
$10^{3}$   & 1000 & k         & Kilo        & Tausend   & thousand\\
$10^{6}$   & ...  & M         & Mega        & Million   & million\\
$10^{9}$   & ...  & G         & Giga        & Milliarde & billion\\
$10^{12}$  & ...  & T         & Tera        & Billion   & trillion\\
$10^{15}$  & ...  & P         & Peta        & Billiarde & quadrillion\\
$10^{18}$  & ...  & E         & Exa         & Trillion  & quintillion\\
\end{tabular}

\begin{tabular}{lrclll}
Potenz     & Zahl & SI-Kürzel & SI-Vorsätze & Deutsch\\
\hline\\
$10^{-1}$  & 0.1   & d         & Dezi        & Zehntel\\
$10^{-2}$  & 0.01  & c         & Centi       & Hundertstel\\
$10^{-3}$  & 0.001 & m         & Milli       & Tausendstel\\
$10^{-6}$  & ...   & $\mu$     & Mikro       & Millionstel\\
$10^{-9}$  & ...   & n         & Nano        & Milliardstel\\
\end{tabular}

Mit dem Taschenrechner können große Zehnerpotenzen mit der
  \GESO{\tiprobutton{EE}}\TALS{\nspirebutton{EE}}-Taste eingegben
  werden: 0.37 Milliarden:

  \GESO{\tiprobutton{0}\tiprobutton{dot}\tiprobutton{3}\tiprobutton{7}\tiprobutton{EE}\tiprobutton{9}}%% END GESO
  \TALS{\nspirebutton{0}\nspirebutton{dot}\nspirebutton{3}\nspirebutton{7}\nspirebutton{EE}\nspirebutton{9}}%% END TALS


\newpage


\subsubsection{Ausklammern von Zehnerpotenzen}
Gegeben ist die folgende Summe. Leider etwas mühsam zum Lesen wegen der vielen Nullen. Klammern Sie Tausend (= $10^3$) aus:

$$400\,000 + 5\,000 + 3\,000\,000 + 70$$
\TNT{3.2}{
$$ = 4\cdot{} 10^5 + 5\cdot{} 10^3 + 3\cdot{} 10^6 + 7\cdot{} 10^1$$
$$ = 10^3 \cdot{} (4\cdot{}10^2 + 5 \cdot{} 1 + 3\cdot{} 10^3 + 7
  \cdot{} 0.01)$$
  $$=10^3\cdot{} (400 + 5 + 3000 + 0.07)  $$
  $$= (3405.07 \cdot{} 10^3) = 3405.07 k$$
}%% END TNT


\paragraph{Ausklammern negativer Zehnerpotenzen:}
\,

\vspace{1mm}

Genauso, wie man positive Zehnerpotenzen ausklammern kann, kann man auch negative Zehnerpotenzen ausklammern. Dies ist insofern praktisch, um sich einen Überblick zu verschaffen, bei sehr kleinen positiven Größen:

Klammern Sie einen Millionstel (=$10^{-6}$) aus:


$$a\cdot{}10^{-6} + b\cdot{}10^{-2} + c\cdot{}10^{-5} +
d\cdot{}10^{-1} = \LoesungsRaumLang{(a + b\cdot{}10^4 + 10c + d\cdot{}10^{5}) \cdot{} 10^{-6}}$$

\subsection*{Aufgaben}

\TALS{Zehnerpotenzen:}
\TALSAadBMTA{41ff}{110. a) b) c) d), 111. c) f), 112. a) d) e) f) m) 114. und 116.}
\TALS{Zehnerpotenzen ausklammern:}
\TALSAadBMTA{34}{88. e) und h)}

\GESO{
  \GESOAadBMTA{65ff}{3. b) c), 4. c), 12. und 16. a) und c)}
  \GESOAadBMTA{68ff}{24. c), 28. c)}
  \GESOAadBMTA{74ff}{52. a) c) d), 53. a) b) d) h) i), 57. a) b) e), 60., 63. und 64.}
}%% END GESO
\newpage


\newcommand{\aaaa}{a\cdot a \cdot a \cdot{} ... \cdot a}
\newcommand{\bbbb}{b\cdot b \cdot b \cdot{} ... \cdot b}

\newpage
\subsection{Potenzen, Definitionen und Gesetze}\index{Potenzen}

\subsubsection{Einführungsbeispiele}
Die sieben verschiedenen schweizer Münzen werden nacheinander geworfen
und es wird notiert, welcher Wurf Zahl oder «Kopf» aufweist. Dieses
Experiment wird einige Male wiederholt, bis man sich die Frage stellt:
Wie viele mögliche Ausgänge hat das Experiment?

\TNT{6}{$$2^7 = 128$$\vspace{22mm}}

Berechen Sie
$$5^3$$

\TNT{4}{125\vspace{16mm}}

Vereinfachen Sie:

$$a^5\cdot(ab)^3\cdot{}(2c)^{2+3}$$

\TNTeop{$$32a^8b^3c^5$$\vspace{22mm}}

%%%%%%%%%%%%%%%%%%%%%%%%%%%%%%%%%%%%%%%%%%%%%%%%%%%%%%%%%

\begin{definition}{Potenz}{}\index{Potenz}
Unter der $n$-ten \textbf{Potenz} verstehen wir eine $n$-malige Multiplikaiton mit demselben Faktor:
$$a^2 = a\cdot a$$
$$a^n = \underbrace{\aaaa}_{n\, \textrm{Faktoren}}$$
\end{definition}
Dabei bezeichnen

\bbwCenterGraphic{9cm}{allg/alg/potenzen_wurzeln/img/Potenzbegriff.png}

%\begin{itemize}
% \item Potenz\index{Potenz}: $a^n$
% \item Exponent\index{Exponent}: $n$
% \item Basis\index{Basis}: $a$
%\end{itemize}


\subsubsection{gleiche Basis}\index{Potenzen!Gesetze}

\begin{itemize}
 \item \fbox{$a^m\cdot a^n = a^{m+n}$}

   Begründung:

   \TNT{2.4}{
   \TALS{$a^m\cdot a^n =
    \underbrace{{\underbrace{\aaaa}_{m\textrm{-mal}}}\cdot{\underbrace{\aaaa}_{n\textrm{-mal}}}}_{m+n\textrm{-mal}}
    = a^{m+n}$}
   \GESO{$7^3\cdot{} 7^5 = (7\cdot{}7\cdot{}7)\cdot{}(7\cdot{}7\cdot{}7\cdot{}7\cdot{}7)=7^8$}%%
}%% END TNT

   
   \item \fbox{$a^m : a^n = \frac{a^m}{a^n}= a^{m-n}$}
     
     Begründung:

     \TNT{2.4}{%
     \TALS{$a^m :  a^n =
    \underbrace{{\underbrace{\aaaa}_{m\textrm{-mal}}} : {\underbrace{\aaaa}_{n\textrm{-mal}}}}_{m-n\textrm{-mal}}
    = a^{m-n}$}
     \GESO{
  Kürzen: 
   $7^5 :  7^3 = \frac{7^5}{7^3} =
  \frac{7\cdot{}7\cdot{}7\cdot{}7\cdot{}7}{7\cdot{}7\cdot{}7} = 7^2 =
  7^{5-3}$}
     }%% END TNT

  
\item \fbox{$(a^n)^m = a^{n\cdot{}m} = (a^m)^n$}

  Begründung:

  \TNT{2.4}{Beispiel $(a^4)^3$ vorrechnen \vspace{12mm}}%%
  
\end{itemize}
\newpage

%%%%%%%%%%%%%%%%%%%%%%%%%%%%%%%%%%%%%%%%%%%%%%%%%%%%%%%%%%%

\subsubsection{gleiche Exponenten}

\begin{itemize}
\item \fbox{$a^n \cdot{} b^n  = (ab)^n$}
  
  Begründung

  \TNT{2.4}{
    \TALS{$a^n\cdot b^n
= \underbrace{\aaaa}_{n\textrm{-mal}}\underbrace{\bbbb}_{n\textrm{-mal}}
= \underbrace{ab\cdot ab\cdot ... \cdot ab}_{n\textrm{-mal}} =
(ab)^n$}%%
   \GESO{\zB $7^3 \cdot 5^3 = (7\cdot{}7\cdot{}7) \cdot{}
     (5\cdot{}5\cdot{}5) = (7\cdot{}5)\cdot{} (7\cdot{}5)\cdot{} (7\cdot{}5) =(7\cdot{}5)^3$}
}%% end TRAINER
   
\item \fbox{$a^n : b^n = (a:b)^n$} bzw. \fbox{$\frac{a^n}{b^n} = \left(\frac{a}{b}\right)^n$} (Gilt ganz analog.)
\end{itemize}

%%%%%%%%%%%%%%%%%%%%%%%%%%%%%%%%%%%%%%%%%%%%%%%%%%%%%%%%%%%%%%

\textbf{Aber Vorsicht}

$$a^3 \cdot{}  b^4 \ne \, ???$$ \TRAINER{(\textrm{Keine gleiche Basis, kein
  gleicher Exponent})}
$$a^3 +        b^3 \ne \, ???$$ \TRAINER{ (\textrm{keine Punkt-Rechunung!})}
\newpage

\subsubsection{Theorieaufgaben}

Lösen Sie:

%\TRAINER{
%$a^4 \cdot a^5 = (a\cdot a\cdot a\cdot a) \cdot (a\cdot a\cdot a\cdot a\cdot a) = a^9 = a^{4+5}$}%%

 $-a^4\cdot (-a)^5 =\LoesungsRaum{-(a\cdot a\cdot a\cdot a) \cdot ((-a)\cdot (-a)\cdot (-a)\cdot (-a)\cdot (-a)) = +a^9}$

 $r^6 : r^4 =\LoesungsRaum{r^{6-4}=r^2}$

 $a^4\cdot b^4 =\LoesungsRaum{(a\cdot a\cdot a\cdot a) \cdot (b\cdot b\cdot b\cdot b) = (a\cdot{}b)^4 = (ab)^4}$

$p^5 : q^5 = \LoesungsRaum{(p:q)^5=\left(\frac{p}{q}\right)^5}$

$(-s^4)^3 = \LoesungsRaum{-s^{4\cdot{}3} = -s^{12}}$

Exponentenvergleich: Finde $x$:
$(r^x)^{10}\cdot{}r^{22} = r^{72}$, $x=\LoesungsRaum{5}$



\subsection*{Aufgaben}
\GESOAadBMTA{66ff}{5. a) c) und 6. a) b)}


\GESO{Gleiche Basis:}

\GESOAadBMTA{69ff}{25. b) f), 29. b) f), 30. b) c) d) e), 33. e) f)}


\GESO{Potenzen von Potenzen:}

\GESOAadBMTA{70ff}{39. b) d) e)}


\GESO{Gleiche Exponenten:}

\GESOAadBMTA{70ff}{40. a) c) f), 42. f) g) h) i)}


\GESO{Erste Exponentialgleichungen:}

\GESOAadBMTA{71}{44. a) b) c)}

%%  OLAT Arbeitsblatt
\GESO{\olatLinkArbeitsblatt{Potenzgesetze}{https://olat.bbw.ch/auth/RepositoryEntry/572162163/CourseNode/102690264435484}{Kapitel 1 und 2}}%% END olatLinkArbeitsblatt
\TALS{\olatLinkArbeitsblatt{Potenzgesetze}{https://olat.bbw.ch/auth/RepositoryEntry/572162090/CourseNode/104915210426569}{Kapitel 1 und 2}}%% END olatLinkArbeitsblatt


\TALS{Aufgaben noch ohne negative Exponenten:}
\TALSAadBMTA{32ff}{79. a), 90. d) l), 91. l), 92. a) d), 93. j), 94. c) f), 95. a),
  101. a), 103. a) b) c), 105. a) g) h) und 106. f)}
\newpage


\subsubsection{Negative Exponenten}
\sectuntertitel{Nicht für alle ist die Potenzrechnung \textit{positiv}.}
Wir kennen bereits das Rechengesetz für positive Exponenten:

$$a^5\cdot{}a^2 = a^{5+2} = a^7$$

Sinnvoll wäre die folgende Erweiterung auf negative
Exponenten:\\\TRAINER{Damit die Rechengesetze weiterhin gelten.}
$$a^5\cdot{}a^{-2} = a^{5+(-2)} = a^3$$

Dividieren wir obige Gleichung beidseitig durch $a^5$, so erhalten wir
folgende sinnvolle Definition:

\TNT{3.2}{
Wir wollen erreichen, dass gilt: $a^5\cdot{}a^{-2} \stackrel{!}{=}a^3$

Was wäre eine sinnvolle Definition für $a^{-2}$, damit die
Rechengesetze weiterhin gelten?

Wir dividieren durch $a^5$ beidseitig: $a^{-2} = \frac{a^3}{a^5} = \frac{1}{a^2}$

Mit dieser «Herleitung» von $a^{-2}= \frac1{a^2}$ gilt: $a^5\cdot{}a^{-2} = a^5\cdot{}\frac1{a^2} = a^3$
}%% END TNT

\begin{definition}{}{}
$$a^{-n} := \frac{1}{a^n}$$
\end{definition}

\begin{bemerkung}{}{}
$$\frac{1}{a^n}= 1 : \underbrace{a : a : a : ... : a}_{n \textrm{\ Divisoren}}$$
\end{bemerkung}

\begin{gesetz}{}{}
$a^{-n} = \left(\frac1a\right)^n$
\end{gesetz}
Begründung:

\TNTeop{$a^{-n} = \frac1{a^n}
= \frac1{\underbrace{a\cdot{}a\cdot{}a\cdot{}...\cdot{}a}_{n \textrm{\ Faktoren}}}
= \underbrace{\frac1a\cdot{}\frac1a\cdot{}\frac1a\cdot{}...\cdot{}\frac1a}_{n \textrm{\ Faktoren}}
= \left(\frac1a\right)^n$
}%% END TNT

%%%%%%%%%%%%%%%%%%%%%%%%%%%%%%%%%%%%%%%%%%%%%%%%%%%%%%

Ganz analog gilt:

\begin{gesetz}{}{}
$\frac{1}{a^{-n}} =a^n$
\end{gesetz}
Begründung:
\TNT{8}{

Beispiel:

$$\frac1{10^{-3}} = \frac1{0.001} = 1000 = 10^3$$

\TALS{TALS:}
Beweis: Definition hinschreiben und auf beiden Seiten den Kehrwert bilden:

$$a^{-n} = \frac1{a^n}$$

$$\frac1{a^{-n}} = a^n$$

\vspace{2cm}
}%% END TNT



\begin{gesetz}{}{}
$\left(\frac{1}{a}\right)^{-n}=a^n$
\end{gesetz}
Begründung:
\TNTeop{
Zahlenbeispiel (erster Schritt nach Definition):
$$\left(\frac1{10}\right)^{-3}  = \frac1{\left(\frac1{10}\right)^3}  = \frac1{0.001} = 1000 = 10^3$$

\TALS{TALS:}

Beweis: Nach Definition gilt für alle $n$:

$$x^{-n} = \frac1{x^n}$$

Somit gilt es auch, wenn wir anstelle von $x$ den Term $\frac1a$ einsetzen:

$$\left(\frac1a\right)^{-n} = \frac1{\frac1{a^n}} = a^n$$


\vspace{2cm}

}%% END TNT

%%%%%%%%%%%%%%%%%%%%%%%%%%%%%%%%%%%%%%


Rechenbeispiel:

Wurm «Wurli» schaft 3 cm pro Sekunde (= $3 \cdot{} 10^{-2} $ m pro Sekunde). Wie lange braucht «Wurli» für 12 m?
\TNT{2.4}{
$$t = \frac{s}v = \frac{12[ \textrm{m}]}{3 \frac{[\textrm{cm}]}{[\textrm{s}]}} =   \frac{12[ \textrm{m}]}{3\cdot{}10^{-2}\frac{[\textrm{m}]}{[\textrm{s}]}} = 4\cdot{} 10^2 [\textrm{s}] = 400 [\textrm{s}]\approx 6-7 Min.$$
}%% END TNT




Und ebenso für beliebige Brüche:

\begin{gesetz}{}{}
$\left(\frac{a}{b}\right)^{-n} = \left(\frac{b}{a}\right)^{+n}$
\end{gesetz}

    
    \TALS{ Begründung

      \TNT{2.4}{$\left( \frac{b}{a} \right)^n  =
       \left(b \cdot{} \frac{1}{a} \right)^n =
       b^n \cdot \left(\frac{1}{a}\right)^n =
       \left(\frac{1}{b}\right)^{-n} \cdot{} a^{-n} =
       \left(\frac{1}{b}\cdot{}a\right)^{-n} =
       \left(\frac{a}{b}\right)^{-n} 
       $}} %% END TNT END TALS

    \GESO{ Begründung

      \TNT{2.4}{$\left( \frac{5}{2} \right)^3  =
       \left(5 \cdot{} \frac{1}{2} \right)^3 =
       5^3 \cdot \left(\frac{1}{2}\right)^3 =
       \left(\frac{1}{5}\right)^{-3} \cdot{} 2^{-3} =
       \left(\frac{1}{5}\cdot{}2\right)^{-3} =
       \left(\frac{2}{5}\right)^{-3} 
       $}} %% END TNT END GESO

%%$$\left(\frac{1}{a}\right)^{-n} = \frac{1}{\left(\frac{1}{a}\right)^n} = a^n$$
\newpage



\subsubsection{Null}

Ebenso müsste $a^5 \cdot a^0 = a^{5+0} = a^5$ gelten. Dies ist aber
nur möglich, wenn wir $a^0 := 1$ definieren ($a\ne 0$).

\begin{definition}{Exponent Null}{} Für alle Basen
$a \in \mathbb{R}\backslash\{0\}$ definieren wir:
\begin{center}
\fbox{$a^0 := 1$}
\end{center}
\end{definition}

Zweite Begründung wenn die Rechengesetze gelten sollten: $a^0 = a^{1-1} = \frac{a^1}{a^1} = 1$

%\textbf{Rechengesetze zusammengefasst:}

%\begin{itemize}
%\item  $\frac{a^m}{a^n} = a^{m-n}$ (Dies gilt auch wenn $n > m$.)
 
%\item $a^{-n} := a^{0-n}=\frac{a^0}{a^n} = \frac{1}{a^n} = \left(\frac{1}{a}\right)^n$

%\item
%$\left(\frac{1}{a}\right)^{-n} = \frac{1}{\left(\frac{1}{a}\right)^n} = a^n$


%\item $\left(\frac{a}{b}\right)^{-n} = \left(\frac{b}{a}\right)^{+n}$ gilt daher auch. 
%\end{itemize}

\begin{rezept*}{«Kielholen»}{}{}
Exponenten vertauschen ihr Vorzeichen beim Übertreten des Bruchstrichs:
$$\frac{a^{-3}b^2}{c^5b^{-6}} = \frac{b^8}{a^3c^5}$$
\end{rezept*}



\subsection{Aufgaben}

\TALS{Potenzen:}\TALSAadBMTA{32ff}{Von Hand: 79. c), 82. a), 83. b),
86. b) c),
91. a), 92. j), 94. k), 96. b) f) h) und 105. i)\\
Prüfen Sie die folgenden Aufgaben auch mit dem TR (Training):\\
103. a) b) c), und 106. h)}

\GESOAadBMTA{67ff}{15., 18. c), 19. b), 20. h), 26. b), 31. b),
  38. c) e), 41. e), 43. c), 44. d) e) f) h) i), 48. a) b), 49. a) c)}

\GESO{\olatLinkArbeitsblatt{Potenzgesetze}{https://olat.bbw.ch/auth/RepositoryEntry/572162163/CourseNode/102690264435484}{Kapitel 3 und 4}}%% END olatLinkArbeitsblatt
\TALS{\olatLinkArbeitsblatt{Potenzgesetze}{https://olat.bbw.ch/auth/RepositoryEntry/572162090/CourseNode/104915210426569}{Kapitel 3 und 4}}%% END olatLinkArbeitsblatt


\GESO{Optional: \GESOAadBMTA{72ff}{51. (Koch)}}

\newpage

