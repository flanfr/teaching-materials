\subsubsection{Doppelbrüche}\index{Doppelbruch}

%%\TALS{S. Buch \cite{frommenwiler17alg} S. 25 Kap. 1.4.4 Dividieren.}
%%\GESO{S. Buch \cite{marthaler21alg} S. 44 Kap. 3.4 Brüche multiplizieren und dividieren}

%\TadBMTA{44/45}{3.4}

\begin{definition}{Doppelbruch}{definition_doppelbruch}\index{Doppelbruch}
  \textbf{Doppelbrüche} sind lediglich eine andere Schreibweise für die
  Division zweier Brüche:\\

  \begin{center}
  \fbox{\huge{$\frac{\frac{a}{b}}{\frac{c}{d}} = \frac{a}{b} :
      \frac{c}{d}$}}
  \end{center}
\end{definition}

\begin{gesetz}{Doppelbruch}{gesetz_doppelbruch}\index{Doppelbruch}
  Brüche werden dividiert, indem man mit dem Kehrwert des Divisors multipliziert:\\

  \begin{center}
  \fbox{\huge{$\frac{\frac{a}{b}}{\frac{c}{d}} = \frac{a}{b} :
      \frac{c}{d} = \frac{a}{b} \cdot\frac{d}{c} = \frac{ad}{bc}$}}
  \end{center}
\end{gesetz}


\begin{beispiel}{}{}
$$\frac{\frac{x+1}{x^2-1}}{\frac{x^2+2x+1}{-2-2x}}$$
\end{beispiel}

\TNTeop{
  $= \frac{x+1}{x^2-1}      :     \frac{x^2+2x+1}{-2-2x}$
  $= \frac{x+1}{(x-1)(x+1)} :     \frac{(x+1)(x+1)}{-2(1+x)}$
  $= \frac{x+1}{(x-1)(x+1)} \cdot \frac{(-2)(1+x)}{(x+1)(x+1)}$
  $= \frac{1}{(x-1)}\cdot\frac{-2}{(x+1)}$
  $= \frac{ -2}{x^2 - 1}$
}
