%% Vektoren in Koordinatensystemen
%% 2020 - 12 - 25 ph. g. freimann @bbw.ch

\section{Koordinatensysteme}



\subsection*{Lernziele}
\begin{itemize}
  \item Vektoren in kartesischen Koordinaten
\item Vektoren in der Polarform
  zeichnen
\item Umrechnen von Kartesischen Koordinaten zu Polarkoordinaten und
  umgekehrt
\item Vektoren in Kartesischen Koordinaten sowie in Polarkoordinaten
  im Taschenrechner 
\end{itemize}
\newpage

\subsection{Kartesische Koordinaten}


\begin{definition}{Kartesische Koordinaten}{}
  Vektoren im Kartesischen Koordinatensystem werden durch ihre $x$-
  bzw. $y$-Komponente angegeben:

  $$\vec{v} = \begin{pmatrix} x_v  \\ y_v \end{pmatrix}$$
\end{definition}

Betrachten Sie nochmals die beiden folgenden (freien) Vektoren ${\color{blue} \vec{a}}$ und
${\color{red}\vec{b}}$:

\bbwGraph{-4}{7}{-3}{3}{
\bbwLetter{3.5,3}{\vec{a}}{blue}
\draw [->,blue] (1,1) -- (4,2);
\bbwLetter{-1,3}{\vec{b}}{red}
\draw [->,red] (-1,1) --(-2,3);
}%% END bbwGraph

Tragen Sie die fehlenden Werte in die Tabelle ein\footnote{Der
  mathematisch positive Winkel wird ab der $x$-Achse im
  Gegenuhrzeigersinn gemessen.}:

\renewcommand{\arraystretch}{2}
\begin{tabular}{|c|c|c|}\hline
                 & ${\color{blue}\vec{a}}$   & ${\color{red}\vec{b}}$   \\\hline
  $x$-Komponente & \TRAINER{3}\noTRAINER{\hspace{10em}}      & \TRAINER{-1}\noTRAINER{\hspace{10em}}   \\\hline
  $y$-Komponente & \TRAINER{1}      & \TRAINER{2}     \\\hline
  Betrag\index{Betrag!eines Vektors} (=Länge) & \TRAINER{$\sqrt{10}$}     & \TRAINER{$\sqrt{5}$}     \\\hline
  math. pos. Winkel  & \TRAINER{$\arctan{}\left(\frac13\right)\approx
    18.43\degre$} & \TRAINER{$90\degre +
    \arctan{}\left(\frac12\right)\approx 116.6\degre$}               \\\hline
\end{tabular}
\renewcommand{\arraystretch}{1}

Koordinatenschreibweise von $\color{blue}\vec{a}$ und $\color{red}\vec{b}$:\,\,
$\vec{a} = \left( \TRAINER{3}\noTRAINER{\,\,\,\,} \atop \TRAINER{1} \right)$
$\vec{b} = \left( \TRAINER{-1}\noTRAINER{\,\,\,\,\,} \atop \TRAINER{2} \right)$
\newpage
\subsubsection{Addition in kartesischen Koordinaten}
Vektoren im kartesischen Koordinatensystem werden addiert, indem ihre
$x$- bzw $y$-Komponenten separat addiert werden:

\begin{gesetz}{Addition}{}\index{Addition!Vektoren}\index{Vektoraddition}
$$\vec{a} + \vec{b} =   \left(x_a \atop x_b \right)  + \left( x_b \atop y_b \right) =
  \left( x_a + x_b \atop y_a + y_b \right)$$
  \end{gesetz}

\begin{beispiel}{Addition}{}
  $$\vec{a} = \left(3\atop 1\right) \textrm{ und } \vec{b} = \left(-1
  \atop 2\right)$$
  $$\Longrightarrow \vec{a} + \vec{b} = \left(3 + (-1) \atop 1 +
  2\right) = \left(2 \atop 3\right)$$
  \end{beispiel}

\platzFuerBerechnungen{6}
\TRAINER{\vspace{6cm}}

\subsection*{Aufgaben}    
    \TALSAadBFWG{188}{46. a) c) und d), 47. und 48.}

\newpage



%% Trigonometrie III
%% Taschenrechner
%% 2020 - 12 - 21 φ@bbw.ch

\section{Taschenrechner}\index{Taschenrechner}

\subsubsection*{Lernziele}

\begin{itemize}
\item Schnittpunkte
\item Graphen im Grad- und Bogenmaß
\end{itemize}

\subsection{Periodische Lösungen einschränken}\index{periodische Lösungen}
Freilich kann mit dem Taschenrechner die Gleichung aus dem
Einstiegsbeispiel $$\cos\left(2x+\frac{\pi}{9}\right)=0.5$$ relativ einfach gelöst werden:

Mit der Eingabe eines Definitionsbereichs (domain) «dom»\footnote{Die Einschränkung auf den Definitionsbereich kann der Taschenrechner rasch ausführen; daher ist dieser hier evtl. bewusst etwas größer gewählt, als im Einstiegsbeispiel.}  ...
$$\mathbb{D} =  [ -\pi , 3\pi]$$
... und der Gleichung «gls» ...
$$\cos\left(2x+\frac{\pi}{9}\right) = 0.5$$

... erhalten wir die selben acht Lösungen.
    
\bbwCenterGraphic{8cm}{tals/trig3/img/solveCos.png}%%

\newpage

\subsection{Referenzaufgabe mit Parametern}
Die im Bogenmaß definierte Funktion $f(x) = a\cdot{}\sin(x)-b$
verlaufe durch die Punkte $P=(2.4\cdot{}\pi | 0.2)$ und
$Q=(2.3\cdot{}\pi|0.4)$. Bestimmen Sie die Parameter $a$ und $b$.
\TNT{8}{
  \bbwCenterGraphic{8cm}{tals/trig3/img/referenzaufgabeTR.png}
}

\newpage


\subsection{Maturaaufgabe 2019}\index{Tageslänge}
(Aus Maturitätsprüfung 2019 Serie 1 Teil 2 Aufgabe 6.)

Die Zeitspanne zwischen Sonnenauf\index{Sonnenaufgang}- und Sonnenuntergang\index{Sonnenuntergang} verändert sich im Laufe eines Jahres.
In München kann die Zeitspanne $T(x)$ in Stunden am $x$-ten Tag des Jahres näherungsweise durch die Funktion

$$T(x) = a\cdot{}\cos\left( 2\pi\cdot{}\frac{x-172}{365}\right) + c\textrm{, Cosinusfunktion im Bogenmass}$$

modelliert werden.

Am 81. Tag misst die Zeitspanne 12 Stunden und am 249. Tag 13 Stunden.

a) Berechnen Sie die Koeffizienten $a$ und $c$ der Funktionsgleichung.

b)  Berechnen Sie $x$ so, dass die Zeitspanne 14 Stunden misst.

c)  Erstellen Sie eine qualitative Skizze für 1 Jahr à 365 Tage.

\TNT{14.8}{
  Vorbereitung im Taschenrechner: Definiere $T(x)$:
  $$T(x) := a\cdot{}\cos\left(2\pi\cdot{}\frac{x-172}{365}\right) + c$$

  a) Löse das Gleichungssystem mittels \texttt{solve(gls, \{a, c\})}:

  \gleichungZZ{T(81)}{12}{T(249)}{13}
  
   Dies liefert $a\approx{}4.192$ und $c\approx{}11.98$

  b) Füge dem Gleichungssystem $T(x)=14$ als Gleichung hinzu und schränke den Definitionsbereich mit $|$ ein:
    $$\textrm{\texttt{solve}}(\{T(81)=12;T(249)=13;T(x)=14\},\{a,c,x\})|x>0 \textrm{ AND } x<365$$
    Dies liefert für $x\approx{110.0}$ oder $x=234.1$.

    c) $x$-Achse von 0 bis 365 und $y$-Achse = $T(x)$; hier die vier bekannten Punkte eintragen.
    Lösung s. Maturaprüfung 2019 Serie 1 Teil 2 Aufgabe 6.
}%% END TNT
\newpage

\subsection{Maturaaufgabe 2020}
(Aus der Maturaprüfung Grundlagenfach 2020 Januar Teil 2 Aufgabe 1.)

Der Graph der Funktion $f: y=3\cdot{}\sin(2x+5)$ ($x$ im Bogenmass)  wird von der Geraden $g$ in den beiden Punkten $A=(-2|...)$ und $B=(2|...)$ geschnitten.

a) Berechnen Sie die $y$-Koordinaten der Schnittpunkte $A$ und $B$.

b) Berechnen Sie den Steigungswinkel der Geraden $g$.


c) $f$ hat im Intervall $[0; \pi]$ die beiden Nullstellen $x_1$ und $x_2$. Berechnen Sie die Funktionsgleichung der Parabel $p$ mit den Nullstellen $x_1$ und $x_2$, die die $y$-Achse bei $y=10$ schneidet. Geben Sie $p$ in der Nullstellenform an.

\TNT{16}{
  Vorbereitung neues «Problem» im Taschenrechner

  $$f(x) := 3\cdot{} \sin(2x+5)$$
  
  \textbf{a)} (Braucht die Gerade noch nicht)
  Berechne (im Bogenmaß) $f(-2) = 3\cdot{}\sin(1) \approx 2.524$ und $f(2) = 3\cdot{} \sin(9) \approx 1.236$

  \textbf{b)} Definiere
  $$g(x) := a\cdot{}x+b$$
  $$gls:=\{ f(-2)=g(-2)  ; f(2) = g(2)\}$$
  $$ \textrm{solve} (gls;\{a, b\})$$

  Dies liefert die Steigung $a$ = 

  $a = \frac{3\cdot{}\sin(9) -\sin(1)}{4} (\approx{} -0.3220)$ (Winkel im Bogenmaß)!

  Somit ist der \textbf{Steigungswinkel} $\varphi = \arctan(a)$:

  rad: $\varphi = \arctan\left(\frac{3\cdot{}(\sin(9)-\sin(1))}4\right) \approx -0.3115 \textrm{ rad }$

  Grad: $\varphi = \arctan\left(\frac{3\cdot{}\left(\sin\left(\frac{9\cdot{}180}{\pi}\right)-\sin\left(\frac{1\cdot{}180}{\pi}\right)\right)}4\right) \approx -17.85\degre$


  \textbf{c)} Die Nullstellen werden typischerweise mit $\textrm{\texttt{solve}} (f(x)=0, x)$ gelöst:

  $$x_1\approx{} 0.6416 \textrm{ und } x_2\approx{} 2.212$$
  Somit lautet die Funktionsgleichung in Nullstellenform:

  $$f(x) = a(x-x_1)\cdot{}(x-x_2)\approx{} a(x-0.6416)\cdot{}(x-2.212)$$
  Das $a$ kann mit folgender Gleichung gefunden werden:
  $$p(x) := a\cdot{}(x-0.6416)\cdot{}(x-2.212); \textrm{\texttt{solve}} (p(0)=10; a); a\approx{}7.045$$
}%% END TNT
\newpage

\newpage

\subsection{Länge im kartesischen Koordinatensystem}

Die Länge der Vektoren wird mittels «Pythagoras» berechnet.

Sei $\vec{a}  = \begin{pmatrix}x_a\\y_a\end{pmatrix}$. Somit ist die Länge von
    $\vec{a}$ wie folgt zu berechnen:

    \begin{gesetz}{Betrag, Länge}{}
      Betrag von $\vec{a}$ := Länge von $\vec{a}$

      $$a = |\vec{a}| = \sqrt{x_a^2 + y_a^2}$$
      \end{gesetz}
    Notationen:

    \begin{beispiel}{}{}
      $$ \vec{a}= \begin{pmatrix} 3\\ 1\end{pmatrix}$$
        $$|\vec{a}| = \LoesungsRaumLang{\sqrt{3^2+1^2} = \sqrt{10}\approx 3.162}$$
      \end{beispiel}
    

    \begin{bemerkung}{}{}
      Im Taschenrechner werden die Vektoren mit eckigen Klammern
      definiert:

      \texttt{a := [3; 1]}

      Die Länge (Betrag) wird mit dem Befehl \texttt{norm} ermittelt:

      \texttt{norm(a)}
    \end{bemerkung}

\newpage

\subsection{Polarkoordinaten}\index{Polarkoordinaten}
\sectuntertitel{Opposite of a polarbear? A cartesian bear!}

\bbwCenterGraphic{16cm}{tals/vecg1/img/polen.jpg}

\newpage

Anstelle der $x$- und der $y$-Komponente können wir mit dem selben
Informationsgehalt auch den Winkel ($\varphi$)
(in mathematisch positiver Richtung) und die Länge ($r$) eines Vektors
angeben.

\bbwCenterGraphic{10cm}{tals/vecg1/img/polar.png}

Dieser Winkel wird im mathematisch positiven Sinne ab der $x$-Achse
angegeben. Der Vektor
$\begin{pmatrix}0\\1\end{pmatrix}$ hat somit den Winkel $90\degre$.
  \begin{definition}{Polarkoordinaten}{}
    Vektoren in Polarkoordinaten werden in der Reihenfolge (Länge |
    Winkel) angegeben:
    $$\vec{a} = (r | \varphi)$$
    Dabei ist $r$ die Länge und $\varphi$ der Winkel im
    Gegenuhrzeigersinn ab der $x$-Achse gemessen.
  \end{definition}

  \begin{beispiel}{Polarkoorinaten}{}
    $$\vec{a} = (2 | 60\degre) = \LoesungsRaum{\left(1\atop \sqrt{3}\right)}$$
  \end{beispiel}
  
  \begin{bemerkung}{Nullvektor}{}\index{Nullvektor}
    Der Nullvektor hat keine Richtung.
    \end{bemerkung}
  \newpage
\subsection{Transformation}\index{Transformation!Polar-
    vs. Kartesische Koordinaten}
  Die Umrechnung von Polar- zu kartesischen
  Koordinaten und umgekehrt wird Transformation genannt.
  
  Dank unseren Freunden \textit{Sinus} und \textit{Cosinus} ist die
  Transformation aus Polarkoordinaten relativ einfach:
  \begin{rezept}{Transformation «polar» nach «kartesisch»}{}

    Gegeben:  Winkel $\varphi$ und Länge $r$

    Gesucht: $x$ und $y$ in kartesischen Koordinaten

    $$x = r\cdot{}\cos(\varphi)$$
    $$y = r\cdot{}\sin(\varphi)$$
  \end{rezept}

  Die Umkehrung ist etwas komplizierter, denn dabei müssen wir
  beachten, in welchen Quadranten die Pfeilspitze des Ortsvektors
  zeigt.
  Zum Glück nimmt uns das der Taschenrechner ab.

  \subsubsection{Kartesische Koordinaten aus Polarkoordinaten}
  Der Taschenrechner verwandelt Polarkoordinaten automatisch immer in
  kartesische Koordinaten um:

    \bbwCenterGraphic{5cm}{tals/vecg1/img/Polar2Kartesisch.png}
    \newpage

    
  \subsubsection{Polarkoordinaten aus kartesischen}
  \begin{beispiel}{kartesisch zu polar}{}
    Gegeben der Vektor $\vec{a} = \left(-\sqrt{3} \atop 1\right)$:

    Eine Skizze zeigt uns rasch, dass es sich um $150\degre$ und eine
    Länge von $2$ handeln muss.

    \bbwCenterGraphic{5cm}{tals/vecg1/img/Kartesisch2Polar.png}
    \end{beispiel}

Rechnen Sie in kartesische Koordinaten um:

\begin{tabular}{rcl}
  $(2   | 30\degre)$ & $=$ & \LoesungsRaum{$\left(\sqrt3\atop 1\right)$}\\
  $(321 | 270\degre)$ & $=$ & \LoesungsRaum{$\left(0\atop -321\right)$}\\
  $(0   | 103.83346\degre)$ & $=$ & \LoesungsRaum{$\left(0\atop 0\right)$}
\end{tabular}

Rechnen Sie in Polarkoordinaten um:

\begin{tabular}{rcl}\vspace{2mm}
  $\left(2\atop 2 \right)$ & $=$ & \LoesungsRaum{$(2\cdot{}\sqrt2|45\degre)$}\\\vspace{2mm}
  $\left(0\atop -13 \right)$ & $=$ & \LoesungsRaum{$(13|270\degre)$}\\\vspace{2mm}
  $\left(-5\atop 1 \right)$ & $=$ & \LoesungsRaum{$(\sqrt{26}\cdot{} | {180\degre - \arctan(\frac15)}) \approx (5.099 | 168.69\degre)$}\\\vspace{2mm}
  $\left(0\atop 0 \right)$ & $=$ & \LoesungsRaum{$(0|18.35\degre) = (0|299.68\degre)= ...$} \\\vspace{2mm}
  
\end{tabular}

\TNT{5.2}{\vspace{5.2cm}}

\newpage
\subsubsection{Drei Schwimmer im Taschenrechner}
Mit dem Taschenrechner wird unsere Einstiegs-Aufgabe mit den drei
Schwimmern stark vereinfacht. Wir geben die Geschwindigkeit und
Richtung des Schwimmers «A» in Polarkoordinaten an
(\textbf{\texttt{schwimmera}}).

Die Geschwindigkeit und Richtung des Flusses können wir auch in kartesischen
Koordinaten angeben (\texttt{\textbf{fluss}}).

Der resultierende Vektor aus Schwimmer und Fluss ist einfach
die Summe der beiden Vektoren.

Wir suchen nun also ein Vielfaches ($t$) der resultierenden
Geschwindigkeit, sodass in $y$-Richtung gerade die 0.2 km erreicht
wird. Die «Verschiebung» am anderen Ufer in $x$-Richtung bezeichnen
wir einfach mit $x$.

\bbwCenterGraphic{12cm}{tals/vecg1/img/TR_DreiSchwimmer.png}

\begin{bemerkung}{}{}
  Auch wenn es nur eine Gleichung mit zwei Unbekannten ist, dürfen wir
  nicht vergessen, das Vektoren in der Ebene aus zwei Komponenten
  bestehen, somit haben wir genau genommen auch ein lineares
  Gleichungssystem mit zwei Gleichungen und zwei Unbekannten zu lösen.
  \end{bemerkung}
\newpage
  
\subsection*{Aufgaben}
\TALSAadBMTA{181ff}{18., 21. und 24.}

Mit Taschenrechner:

\TALSAadBMTA{181ff}{23.}
\newpage
